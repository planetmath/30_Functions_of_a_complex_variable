\documentclass[12pt]{article}
\usepackage{pmmeta}
\pmcanonicalname{HolomorphicFunctionAssociatedWithContinuousFunction}
\pmcreated{2013-03-22 19:14:29}
\pmmodified{2013-03-22 19:14:29}
\pmowner{pahio}{2872}
\pmmodifier{pahio}{2872}
\pmtitle{holomorphic function associated with continuous function}
\pmrecord{11}{42166}
\pmprivacy{1}
\pmauthor{pahio}{2872}
\pmtype{Theorem}
\pmcomment{trigger rebuild}
\pmclassification{msc}{30E20}
\pmclassification{msc}{30D20}
\pmrelated{DifferentiationUnderIntegralSign}
\pmrelated{CauchyIntegralFormula}

% this is the default PlanetMath preamble.  as your knowledge
% of TeX increases, you will probably want to edit this, but
% it should be fine as is for beginners.

% almost certainly you want these
\usepackage{amssymb}
\usepackage{amsmath}
\usepackage{amsfonts}

% used for TeXing text within eps files
%\usepackage{psfrag}
% need this for including graphics (\includegraphics)
%\usepackage{graphicx}
% for neatly defining theorems and propositions
 \usepackage{amsthm}
% making logically defined graphics
%%%\usepackage{xypic}

% there are many more packages, add them here as you need them

% define commands here

\theoremstyle{definition}
\newtheorem*{thmplain}{Theorem}

\begin{document}
\textbf{Theorem.}\, If $f(z)$ is continuous on a (finite) contour $\gamma$ of the complex plane, then the contour integral
\begin{align}
g(z) \;=:\; \int_{\gamma}\frac{f(t)}{t\!-\!z}\,dt,
\end{align}
defines a function \,$z \mapsto g(z)$\, which is holomorphic in any domain $D$ not containing points of $\gamma$.\, Moreover, the derivative has the expression
\begin{align}
g'(z) \;=\; \int_{\gamma}\!\frac{f(t)}{(t\!-\!z)^2}\,dt.
\end{align}



\emph{Proof.}\, The right hand side of (2) is defined since its integrand is continuous.\, On has to show that it equals
$$\lim_{\Delta z \to 0}\frac{g(z\!+\!\Delta z)\!-\!g(z)}{\Delta z}.$$
Let\, $z_1 =: z\!+\!\Delta z \notin \gamma$,\; $\Delta z \neq 0$.\, We may write first
$$\frac{g(z_1)-g(z)}{z_1\!-\!z} 
\;=\; \frac{1}{\Delta z}\int_{\gamma}f(t)\left[\frac{1}{t\!-\!z_1}-\frac{1}{t\!-\!z}\right]\,dt
\;=\; \int_{\gamma}\frac{f(t)}{(t\!-\!z_1)(t\!-\!z)}\,dt,$$
whence
$$E \;=:\; \frac{g(z_1)-g(z)}{z_1\!-\!z}-\int_{\gamma}\frac{f(t)}{(t\!-\!z)^2} 
\;=\; \Delta z\cdot\!\int_{\gamma}\frac{f(t)}{(t\!-\!z_1)(t\!-\!z)^2}\,dt.$$
Because $f$ is continuous in the compact set $\gamma$, there is a positive constant $M$ such that
$$|f(t)| \;<\; M \quad \forall\; t \in \gamma.$$
As well, we have a positive constant $d$ such that
$$|t\!-\!z| \;\geqq\; d \quad \forall\; t \in \gamma.$$
When we choose\, $|\Delta z| < \frac{d}{2}$,\, it follows that
$$|t\!-\!z_1| \;=\; |(t\!-\!z)-\Delta z| \;\geqq\; |t\!-\!z|-|\Delta z| \;>\; d\!-\!\frac{d}{2} \;=\; \frac{d}{2}.$$
Consequently, 
$$\left|\frac{f(t)}{(t\!-\!z_1)(t\!-\!z)^2}\right| \;=\; \frac{|f(t)|}{|t\!-\!z_1||t\!-\!z|^2} 
\;<\; \frac{M}{\frac{d}{2}\cdot d^2} \;=\; \frac{2M}{d^3}$$
and, by the estimating theorem of contour integral, 
$$|E| \;=\; |\Delta z|\cdot\left|\int_{\gamma}\frac{f(t)}{(t\!-\!z_1)(t\!-\!z)^2}\,dt\right|
\;<\; |\Delta z|\cdot\frac{2M}{d^3}\cdot k,$$
where $k$ is the length of the contour.\, The last expression tends to zero as\, $\Delta z \to 0$.\, This settles the proof.\\


\textbf{Remark 1.}\, By induction, one can prove the following generalisation of (2):
\begin{align}
g^{(n)}(z) \;=\; n!\!\int_{\gamma}\!\frac{f(t)}{(t\!-\!z)^{n+1}}\,dt \qquad (n \;=\; 0,\,1,\,2,\,\ldots)
\end{align}

\textbf{Remark 2.}\, The contour $\gamma$ may be \PMlinkescapetext{open or closed}.\, If it especially is a circle, then (1) defines a holomorphic function inside $\gamma$ and another outside it.

%%%%%
%%%%%
\end{document}
