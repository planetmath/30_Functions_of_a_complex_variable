\documentclass[12pt]{article}
\usepackage{pmmeta}
\pmcanonicalname{ExistenceOfPowerSeries}
\pmcreated{2013-03-22 12:56:27}
\pmmodified{2013-03-22 12:56:27}
\pmowner{rmilson}{146}
\pmmodifier{rmilson}{146}
\pmtitle{existence of power series}
\pmrecord{5}{33298}
\pmprivacy{1}
\pmauthor{rmilson}{146}
\pmtype{Result}
\pmcomment{trigger rebuild}
\pmclassification{msc}{30B10}

\endmetadata

\usepackage{amsmath}
\usepackage{amsfonts}
\usepackage{amssymb}
\newcommand{\reals}{\mathbb{R}}
\newcommand{\natnums}{\mathbb{N}}
\newcommand{\cnums}{\mathbb{C}}
\newcommand{\znums}{\mathbb{Z}}
\newcommand{\lp}{\left(}
\newcommand{\rp}{\right)}
\newcommand{\lb}{\left[}
\newcommand{\rb}{\right]}
\newcommand{\supth}{^{\text{th}}}
\newtheorem{proposition}{Proposition}
\newtheorem{definition}[proposition]{Definition}

\newtheorem{theorem}[proposition]{Theorem}
\begin{document}
In this entry we shall demonstrate the logical equivalence of the holomorphic
and analytic concepts.   As is the case with so many basic results in
complex analysis, the proof of these facts hinges on the Cauchy
integral theorem, and the Cauchy integral formula.

\paragraph{Holomorphic implies analytic.}
\begin{theorem}
Let $U\subset\cnums$ be an open domain that contains the origin, and
let $f:U\rightarrow\cnums,$
be a function such that
the complex derivative
$$f'(z) = \lim_{\zeta\rightarrow 0} \frac{f(z+\zeta)-f(z)}{\zeta}$$
exists for all $z\in U$.  Then,  there exists a power series representation
$$f(z) = \sum_{k=0}^\infty a_k z^k,\quad \Vert z\Vert < R,\quad
a_k\in\cnums$$ 
for a
sufficiently small radius of convergence $R>0$.  
\end{theorem}
Note: it is just as
easy to show the existence of a power series representation around
every basepoint in $z_0\in U$; one need only consider the holomorphic
function $f(z-z_0)$.

\emph{Proof.}  Choose an $R>0$ sufficiently small so that the 
disk $\Vert z\Vert \leq R$ is contained in $U$.
By the Cauchy integral formula we have
that
$$f(z) = \frac{1}{2\pi i} \oint_{\Vert \zeta \Vert = R} \frac{f(\zeta)}{\zeta-z}\,
d\zeta,\quad \Vert z\Vert < R,$$
where, as usual, the integration contour is oriented
counterclockwise.  For every $\zeta$  of modulus $R$, we can expand
the integrand as a geometric power series in $z$, namely
$$\frac{f(\zeta)}{\zeta-z} = \frac{f(\zeta)/\zeta}{1-z/\zeta} =
\sum_{k=0}^\infty\, \frac{f(\zeta)}{\zeta^{k+1}}\, z^k,\quad \Vert
z\Vert < R.$$
The circle of radius $R$ is a compact set; hence
$f(\zeta)$ is bounded on it; and hence, the power series above
converges uniformly with respect to $\zeta$.  Consequently, the order
of the infinite summation and the integration operations can be
interchanged. Hence,
$$f(z) = \sum_{k=0}^\infty a_k z^k,\quad \Vert z\Vert< R,$$
where
$$a_k = \frac{1}{2\pi i} \oint_{\Vert \zeta \Vert = R}
\frac{f(\zeta)}{\zeta^{k+1}},$$
as desired.  QED

\paragraph{Analytic implies holomorphic.}
\begin{theorem}
Let 
$$f(z) = \sum_{n=0}^\infty a_n z^n,\quad a_n\in\cnums,\quad \Vert
z\Vert<\epsilon$$ 
be a power series, converging in $D=D_\epsilon(0)$, the open disk of radius
$\epsilon>0$ about the origin.  Then the complex derivative
$$f'(z) =\lim_{\zeta\rightarrow 0} \frac{f(z+\zeta)-f(z)}{\zeta}$$
exists for all $z\in D$, i.e. the function $f:D\rightarrow
\cnums$ is holomorphic.
\end{theorem}
Note: this theorem generalizes immediately to shifted power series in
$z-z_0,\; z_0\in\cnums$.

\emph{Proof.} For every $z_0\in D$, the function $f(z)$ can be recast
as a power series centered at $z_0$.  Hence, without loss of
generality it suffices to prove the theorem for $z=0$.  The power
series
$$\sum_{n=0}^\infty a_{n+1} \zeta^n,\quad \zeta\in D$$
converges, and equals
$(f(\zeta)-f(0))/\zeta$ for $\zeta\neq 0$. Consequently, the complex
derivative $f'(0)$ exists; indeed it is equal to $a_1$.  QED
%%%%%
%%%%%
\end{document}
