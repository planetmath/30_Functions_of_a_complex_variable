\documentclass[12pt]{article}
\usepackage{pmmeta}
\pmcanonicalname{TopicEntryOnComplexAnalysis}
\pmcreated{2013-05-20 18:11:35}
\pmmodified{2013-05-20 18:11:35}
\pmowner{pahio}{2872}
\pmmodifier{unlord}{1}
\pmtitle{topic entry on complex analysis}
\pmrecord{60}{39009}
\pmprivacy{1}
\pmauthor{pahio}{1}
\pmtype{Topic}
\pmcomment{trigger rebuild}
\pmclassification{msc}{30A99}
\pmrelated{HarmonicConjugateFunction}
\pmrelated{TakingSquareRootAlgebraically}
\pmrelated{CalculatingTheNthRootsOfAComplexNumber}
\pmrelated{FundamentalTheoremOfAlgebra}
\pmrelated{FundamentalTheoremsInComplexAnalysis}
\pmdefines{complex analytic}

\endmetadata

% this is the default PlanetMath preamble.  as your knowledge
% of TeX increases, you will probably want to edit this, but
% it should be fine as is for beginners.

% almost certainly you want these
\usepackage{amssymb}
\usepackage{amsmath}
\usepackage{amsfonts}

% used for TeXing text within eps files
%\usepackage{psfrag}
% need this for including graphics (\includegraphics)
%\usepackage{graphicx}
% for neatly defining theorems and propositions
 \usepackage{amsthm}
% making logically defined graphics
%%%\usepackage{xypic}

% there are many more packages, add them here as you need them

% define commands here

\theoremstyle{definition}
\newtheorem*{thmplain}{Theorem}

\begin{document}
\subsection*{Introduction}
Complex analysis may be defined as the study of analytic 
functions of a complex variable.  The origins of this subject
lie in the observation that, given a function which has a
convergent Taylor series, one can substitute complex numbers
for the variable and obtain a convergent series which defines
a function of a complex variable.  Putting
imaginary numbers into the power series for the exponential
function, we find
\begin{eqnarray*}
e^{ix} &=& 1 + i x - \frac{x^2}{2} - i \frac{x^3}{3!} + 
\frac{x^4}{4!} + i \frac{x^5}{5!} - \frac{x^6}{6!} - i 
\frac{x^7}{7!}+\cdots \\
e^{-ix} &=& 1 - i x - \frac{x^2}{2} + i \frac{x^3}{3!} + 
\frac{x^4}{4!} - i \frac{x^5}{5!} - \frac{x^6}{6!} + i 
\frac{x^7}{7!}+\cdots 
\end{eqnarray*}
Adding and subtracting these series, we find
\begin{eqnarray*}
\frac{1}{2} (e^{ix} + e^{-ix}) &=& 1 - \frac{x^2}{2!} + 
\frac{x^4}{4!} - \frac{x^6}{6!} +- \cdots \\
\frac{1}{2i} (e^{ix} - e^{-ix}) &=& x - \frac{x^3}{3!} + 
\frac{x^5}{5!} - \frac{x^7}{7!} +- \cdots
\end{eqnarray*}
We recognize these series as the Taylor-Madhava series for 
the sine and the cosine functions respectively.  We hence have
\begin{eqnarray*}
\sin x &=& \frac{1}{2i} (e^{ix} - e^{-ix}) \\
\cos x &=& \frac{1}{2} (e^{ix} + e^{-ix}) \\
e^{ix} &=& \cos x + i \sin x .
\end{eqnarray*}
These equations let us re-express trigonometric functions
in terms of complex exponentials.  Using them, deriving and
verifying trigonometric identities becomes a straightforward
exercise in algebra using the laws of exponents.

We call functions of a complex variable which can be
expressed in terms of a power series as \emph{complex
analytic}.  More precisely, if $D$ is an open 
subset of $\mathbb{C}$, we say that a function $f
\colon D \to \mathbb{C}$ is complex analytic
if, for every point $w$ in $D$, there exists a positive
number $\delta$ and a sequence of complex numbers
$c_k$ such that the series
\[
\sum_{k=0}^\infty c_k (z-w)^k
\]
converges to $f(z)$ when\, $z \in D$\, and\, 
$|z - w| < \delta$.

An important feature of this definition is that it 
is not required that a single series works for
all points of $D$.  For instance, suppose
we define the function\, $f \colon \mathbb{C}\!\smallsetminus\!\{1\} \to \mathbb{C}$\, as 
\[
f(z) = {1 \over 1 - z}.
\]
While it it turns out that $f$ is analytic, no single
series will give us the values of $f$ for all allowed
values of $z$.  For instance, we have the familiar
geometric series:
\[
f(z) = \sum_{k=0}^\infty z^k
\]
However, this series diverges when\, $|z| > 1$.\,  For
such values of $z$, we need to use other series.
For instance, when $z$ is near $2$, we have the
following series:
\[
f(z) = \sum_{k=0}^\infty (-1)^{k+1} (z-2)^k
\]
This series, however, diverges when\, $|z-2| > 1$.\,
While, for every allowed value of $z$ we can find
some power series which will converge to $f(z)$, no
single power series will converge to $f(z)$ for 
all permissible values of $z$.

It is possible to define the operations of
differentiation and integration for complex
functions.  These operations are well-defined
for analytic functions and have the usual
properties familiar from real analysis.

The class of analytic functions is interesting
to study for at least two main reasons.  Firstly,
many functions which arise in pure and applied
mathematics, such as polynomials, rational functions,
exponential functions. logarithms, trigonometric
functions, and solutions of differential equations
are analytic.  Second, the class of analytic functions
enjoys many remarkable properties which do not hold
for other classes of functions, such as the following:
\begin{description}
\item[Closure]  The class of complex analytic functions 
is closed under the usual algebraic operations,
taking derivative and integrals, composition,
and taking uniform limits.
\item[Rigidity]  Given a complex analytic function\,
$f \colon D \to \mathbb{C}$, where $D$ is
an open subset of $\mathbb{C}$,\, if we know the
values of $f$ at an infinite number of points of
$D$ which have a limit point in $D$, then we 
know the value of $f$ at all points of $D$.  For
instance, given a complex analytic function on
some neighborhood of the real axis, the values
of that function in the whole neighborhood will
be determined by its values on the real axis.
\item[Cauchy and Morera theorems]  The integral of a 
complex analytic function along any contractible
closed loop equals zero.  Conversely, if the integral
of a complex function about every contractible loop 
happens to be zero, then that function is analytic.
\item[Complex differentiability]  If a complex
function is differentiable, then it has derivatives
of all orders.  This contrasts sharply with the
case of real analysis, where a function may be
differentiable only a fixed number of times.
\item[Harmonicity]  The real and imaginary parts
of a complex analytic function are harmonic, i.e.
satisfy Laplace's equation.  Conversely, given a
harmonic function on the plane, there exists a
complex analytic function of which it is the real
part.
\item[Conformal mapping]  A complex function is
analytic if and only if it preserves maps pairs
of intersecting curves into pairs which intersect 
at the same angle.
\end{description}
As one can see, there are many ways to characterize
complex analytic functions, many of which have
nothing to do with power series.  This suggests that
analytic functions are somehow a naturally occurring
subset of complex functions.  This variety of distinct
ways of characterizing analytic functions means that
one has a variety of methods which may be used to
study them and prove deep and surprising results
by bringing insights and techniques from geometry,
differential equations, and functional analysis to
bear on problems of complex analysis.  This also
works the other way --- one can use complex analysis
to prove results in other branches of mathmatics 
which have nothing to do with complex numbers.  For
instance, the problem of minimal surfaces can be
solved by using complex analysis.

\subsection{Complex numbers}
\begin{enumerate}
\item complex plane, equality of complex numbers
\item topology of the complex plane
\item triangle inequality of complex numbers
\item argument of product and quotient
\item unit disc, annulus, closed complex plane
\item \PMlinkname{$n$th root}{CalculatingTheNthRootsOfAComplexNumber}
\item taking square root algebraically
\item quadratic equation in $\mathbb{C}$
\item \PMlinkname{complex function}{ComplexFunction}
\end{enumerate}

\subsection{Complex functions}
\begin{enumerate}
\item de Moivre identity
\item addition formula
\item complex exponential function
\item periodicity of exponential function
\item complex sine and cosine
\item values of complex cosine
\item complex tangent and cotangent
\item example of summation by parts
\item Euler's formulas (see also \PMlinkname{this}{ComplexSineAndCosine})
\item complex logarithm
\item general power
\item fundamental theorems in complex analysis
\item index of special functions
\end{enumerate}

\subsection{Analytic function}
\begin{enumerate}
\item holomorphic
\item meromorphic
\item periodic functions
\item isolated singularity
\item complex derivative
\item Cauchy-Riemann equations
\item \PMlinkname{power series}{PowerSeries}
\item Bohr's theorem
\item identity theorem of holomorphic functions
\item Weierstrass double series theorem
\item entire functions
\item properties of entire functions
\item \PMlinkname{pole of function}{Z_0IsAPoleOfF}
\item zeros and poles of rational function
\item when all singularities are poles
\item Casorati-Weierstrass theorem
\item Picard's theorem
\item Laurent series
\item coefficients of Laurent series
\item residue
\item regular at infinity
\item Nevanlinna theory
\end{enumerate}

\subsection{Complex integration}
\begin{enumerate}
\item contour integral
\item estimating theorem of contour integral 
\item theorems on complex function series
\item holomorphic function associated with continuous function
\item Cauchy integral theorem
\item Cauchy integral formula; variant of Cauchy integral formula
\item \PMlinkname{residue theorem}{CauchyResidueTheorem}
\item example of using residue theorem
\item argument principle
\item complex antiderivative
\end{enumerate}

\subsection{Analytic continuation}
\begin{enumerate}
\item analytic continuation
\item meromorphic continuation
\item analytic continuation by power series
\item monodromy theorem
\item Schwarz' reflection principle
\item example of analytic continuation
\item analytic continuation of gamma function
\item analytic continuation of Riemann zeta to critical strip
\item \PMlinkname{analytic continuation of Riemann zeta (using integral)}{AnalyticContinuationOfRiemannZetaUsingIntegral}
\end{enumerate}

\subsection{Riemann zeta function}
\begin{enumerate}
\item Riemann zeta function
\item Euler product formula
\item \PMlinkname{Riemann functional equation}{FunctionalEquationOfTheRiemannZetaFunction}
\item critical strip
\item \PMlinkid{value of the Riemann zeta function at 0}{8190},\, 
\PMlinkid{at 2}{4719},\, \PMlinkid{at 4}{11009}
\item formulae for zeta in the critical strip
\end{enumerate}

\subsection{Conformal mapping}
\begin{enumerate}
\item conformal mapping
\item conformal mapping theorem
\item simple example of composed conformal mapping
\item example of conformal mapping
\item \PMlinkid{Schwarz--Christoffel transformation}{6289}
\end{enumerate}
%%%%%
%%%%%.
\end{document}
