\documentclass[12pt]{article}
\usepackage{pmmeta}
\pmcanonicalname{ContourIntegral}
\pmcreated{2013-03-22 12:51:44}
\pmmodified{2013-03-22 12:51:44}
\pmowner{Mathprof}{13753}
\pmmodifier{Mathprof}{13753}
\pmtitle{contour integral}
\pmrecord{23}{33198}
\pmprivacy{1}
\pmauthor{Mathprof}{13753}
\pmtype{Definition}
\pmcomment{trigger rebuild}
\pmclassification{msc}{30A99}
\pmclassification{msc}{30E20}
\pmsynonym{complex integral}{ContourIntegral}
\pmsynonym{line integral}{ContourIntegral}
\pmsynonym{curve integral}{ContourIntegral}
\pmrelated{CauchyIntegralFormula}
\pmrelated{PathIntegral}
\pmrelated{Integral}
\pmrelated{IntegralTransform}
\pmrelated{RealAndImaginaryPartsOfContourIntegral}
\pmdefines{contour}

\endmetadata

\usepackage{amssymb}
\usepackage{amsmath}
\usepackage{amsfonts}

\renewcommand{\Im}{\operatorname{Im}}
\begin{document}
\PMlinkescapeword{cover}

Let $f$ be a complex-valued function defined on the image of a \PMlinkname{curve}{Curve} $\alpha$: $[ a, b ] \rightarrow \mathbb{C}$, let $P = \{ a_{0}, ..., a_{n} \}$ be a \PMlinkname{partition}{Partition3} of $[ a, b ]$.  We will restrict our attention to \emph{contours}, i.e. curves for which the parametric equations consist of a finite number of continuously differentiable arcs. 
If the sum

$$\sum_{i = 1}^{n} f(z_{i}) (\alpha(a_{i}) - \alpha(a_{i - 1})),$$

where $z_{i}$ is some point $\alpha(t_{i})$ such that $a_{i - 1} \leqslant t_{i} \leqslant a_{i}$, converges as $n$ tends to infinity and the greatest of the numbers $a_{i} - a_{i - 1}$ tends to zero, then we define the \emph{contour integral} of $f$ along $\alpha$ to be the integral

$$\int_{\alpha} f(z) dz:=\int_a^b f(\alpha(t))d\alpha(t)$$

\section*{Notes} (i) If $\Im(\alpha)$ is a segment of the real axis, then this definition reduces to that of the Riemann integral of $f(x)$ between $\alpha (a)$ and $\alpha (b)$.

(ii) An alternative definition, making use of the Riemann-Stieltjes integral, is based on the fact that the definition of this can be extended without any other changes in the wording to cover the cases where $f$ and $\alpha$ are complex-valued functions.

Now let $\alpha$ be any curve $[a, b] \rightarrow \mathbb{R}^{2}$.  Then $\alpha$ can be expressed in terms of the components $(\alpha_{1}, \alpha_{2})$ and can be associated with the complex-valued function

$$z(t) = \alpha_{1}(t) + i \alpha_{2}(t).$$

Given any complex-valued function of a complex variable, $f$ say, defined on $\Im(\alpha)$ we define the \textbf{contour integral} of $f$ along $\alpha$, denoted by

$$\int_{\alpha} f(z) dz$$

by

$$\int_{\alpha} f(z) dz = \int_{a}^{b} f(z(t)) dz(t)$$
whenever the complex Riemann-Stieltjes integral on the right exists.

(iii) Reversing the direction of the curve changes the sign of the integral.

(iv) The contour integral always exists if $\alpha$ is rectifiable and $f$ is continuous.

(v) If $\alpha$ is piecewise smooth and the contour integral of $f$ along $\alpha$ exists, then

$$\int_{\alpha} f dz = \int_{a}^{b} f(z(t)) z'(t) dt.$$
%%%%%
%%%%%
\end{document}
