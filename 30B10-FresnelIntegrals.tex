\documentclass[12pt]{article}
\usepackage{pmmeta}
\pmcanonicalname{FresnelIntegrals}
\pmcreated{2014-07-11 21:15:59}
\pmmodified{2014-07-11 21:15:59}
\pmowner{pahio}{2872}
\pmmodifier{pahio}{2872}
\pmtitle{Fresnel integrals}
\pmrecord{24}{40992}
\pmprivacy{1}
\pmauthor{pahio}{2872}
\pmtype{Definition}
\pmcomment{trigger rebuild}
\pmclassification{msc}{30B10}
\pmclassification{msc}{26A42}
\pmclassification{msc}{33B20}
%\pmkeywords{Fresnel integral}
\pmrelated{SineIntegral}
\pmdefines{clothoid}

\endmetadata

% this is the default PlanetMath preamble.  as your knowledge
% of TeX increases, you will probably want to edit this, but
% it should be fine as is for beginners.

% almost certainly you want these
\usepackage{amssymb}
\usepackage{amsmath}
\usepackage{amsfonts}
\usepackage{graphicx}

% used for TeXing text within eps files
%\usepackage{psfrag}
% need this for including graphics (\includegraphics)
% for neatly defining theorems and propositions
 \usepackage{amsthm}
% making logically defined graphics
%%%\usepackage{xypic}

% there are many more packages, add them here as you need them

% define commands here

\theoremstyle{definition}
\newtheorem*{thmplain}{Theorem}

\begin{document}
\PMlinkescapeword{expansions}

\subsection{The functions $C$ and $S$}

For any real value of the argument $x$, the {\em Fresnel integrals} $C(x)$ and $S(x)$ are defined as the integrals
\begin{align}
C(x) \;:=\; \int_0^x\cos{t^2}\,dt, \quad\quad
S(x) \;:=\; \int_0^x\sin{t^2}\,dt.
\end{align}

In optics, both of them express the \PMlinkescapetext{intensity of diffracted light behind an illuminated edge}.

Using the Taylor series expansions of \PMlinkname{cosine and sine}{ComplexSineAndCosine}, we get easily the expansions of the functions (1):
$$C(z) \,=\, \frac{z}{1}\!-\!\frac{z^5}{5\!\cdot\!2!}\!
+\!\frac{z^9}{9\!\cdot\!4!}\!-\!\frac{z^{13}}{13\!\cdot\!6!}\!+\!-\ldots$$
$$S(z) \,=\, \frac{z^3}{3\cdot1!}\!-\!\frac{z^7}{7\!\cdot\!3!}\!
+\!\frac{z^{11}}{11\!\cdot\!5!}\!-\!\frac{z^{15}}{15\!\cdot\!7!}\!+\!-\ldots$$
These converge for all complex values $z$ and thus define entire transcendental functions.\\

The Fresnel integrals at infinity have the finite value
$$\lim_{x\to\infty}C(x) \;=\; \lim_{x\to\infty}S(x) \;=\; \frac{\sqrt{2\pi}}{4}.$$

\subsection{Clothoid}

The parametric presentation 
\begin{align}
x \;=\; C(t), \quad\quad y \;=\; S(t)
\end{align}
\PMlinkescapetext{represents} a curve called {\em clothoid}.\, Since the equations (2) both define odd functions, the clothoid has symmetry about the origin.\, The curve has the shape of a ``$\sim$''
(see this \PMlinkexternal{diagram}{http://www.wakkanet.fi/~pahio/A/A/clothoid.png}).

\begin{center}
\includegraphics{Clothoid}
\end{center}

The arc length of the clothoid from the origin to the point \,$(C(t),\,S(t))$\, is simply
$$\int_0^t\sqrt{C'(u)^2+S'(u)^2}\,du = \int_0^t\sqrt{\cos^2(u^2)+\sin^2(u^2)}\,du = \int_0^tdu = t.$$
Thus the \PMlinkescapetext{length} of the whole curve (to the point\, 
$(\frac{\sqrt{2\pi}}{4},\,\frac{\sqrt{2\pi}}{4})$) is infinite.

The \PMlinkname{curvature}{CurvaturePlaneCurve} of the clothoid also is extremely \PMlinkescapetext{simple},
$$\varkappa \,=\, 2t,$$
i.e. \PMlinkname{proportional}{Variation} to the arc lenth; thus in the origin only the curvature is zero.

Conversely, if the curvature of a plane curve varies proportionally to the arc length, the curve is a clothoid.

This property of the curvature of clothoid is utilised in way and railway construction, since the form of the clothoid is very efficient when a \PMlinkescapetext{straight} portion of way must be bent to a turn:\, the zero curvature of the line can be continuously raised to the wished curvature.

%%%%%
%%%%%
\end{document}
