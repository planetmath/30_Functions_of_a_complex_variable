\documentclass[12pt]{article}
\usepackage{pmmeta}
\pmcanonicalname{SchlichtFunctions}
\pmcreated{2013-03-22 14:23:37}
\pmmodified{2013-03-22 14:23:37}
\pmowner{jirka}{4157}
\pmmodifier{jirka}{4157}
\pmtitle{schlicht functions}
\pmrecord{8}{35890}
\pmprivacy{1}
\pmauthor{jirka}{4157}
\pmtype{Definition}
\pmcomment{trigger rebuild}
\pmclassification{msc}{30C45}
\pmsynonym{schlicht function}{SchlichtFunctions}
\pmrelated{KoebeDistortionTheorem}
\pmrelated{Koebe14Theorem}

% this is the default PlanetMath preamble.  as your knowledge
% of TeX increases, you will probably want to edit this, but
% it should be fine as is for beginners.

% almost certainly you want these
\usepackage{amssymb}
\usepackage{amsmath}
\usepackage{amsfonts}

% used for TeXing text within eps files
%\usepackage{psfrag}
% need this for including graphics (\includegraphics)
%\usepackage{graphicx}
% for neatly defining theorems and propositions
\usepackage{amsthm}
% making logically defined graphics
%%%\usepackage{xypic}

% there are many more packages, add them here as you need them

% define commands here
\theoremstyle{theorem}
\newtheorem*{thm}{Theorem}
\newtheorem*{lemma}{Lemma}
\newtheorem*{conj}{Conjecture}
\newtheorem*{cor}{Corollary}
\newtheorem*{example}{Example}
\newtheorem*{prop}{Proposition}
\theoremstyle{definition}
\newtheorem*{defn}{Definition}
\begin{document}
\begin{defn}
The class of univalent functions on the open unit disc in the complex plane
such that for any $f$ in the class we have $f(0) = 0$ and $f'(0) = 1$ is called
the class of {\em schlicht functions}.  Usually this class is denoted by
${\mathcal{S}}$.
\end{defn}

Note that if $g$ is any univalent function on the unit disc, then the function
$f$ defined by
\begin{equation*}
f(z): = \frac{g(z) - g(0)}{g'(0)}
\end{equation*}
belongs to ${\mathcal{S}}$.  So to study univalent functions on the unit disc it  suffices to study ${\mathcal{S}}$.  A basic result on these gives that this set is in fact compact in the space of analytic functions on the unit disc.  

\begin{thm}
Let $\{ f_n \}$ be a sequence of functions in ${\mathcal{S}}$ and
$f_n \to f$ uniformly on compact subsets of the open unit disc.  Then $f$
is in ${\mathcal{S}}$.
\end{thm}

Alternatively this theorem can be stated for all univalent functions by the above remark, but there a sequence of univalent functions can converge either
to a univalent function or to a constant.  The requirement that the first derivative is 1 for functions in ${\mathcal{S}}$ prevents this problem.

\begin{thebibliography}{9}
\bibitem{Conway:complexII}
John~B. Conway.
{\em \PMlinkescapetext{Functions of One Complex Variable II}}.
Springer-Verlag, New York, New York, 1995.
\end{thebibliography}
%%%%%
%%%%%
\end{document}
