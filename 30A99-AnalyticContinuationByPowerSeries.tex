\documentclass[12pt]{article}
\usepackage{pmmeta}
\pmcanonicalname{AnalyticContinuationByPowerSeries}
\pmcreated{2013-03-22 15:41:22}
\pmmodified{2013-03-22 15:41:22}
\pmowner{rspuzio}{6075}
\pmmodifier{rspuzio}{6075}
\pmtitle{analytic continuation by power series}
\pmrecord{4}{37632}
\pmprivacy{1}
\pmauthor{rspuzio}{6075}
\pmtype{Result}
\pmcomment{trigger rebuild}
\pmclassification{msc}{30A99}

% this is the default PlanetMath preamble.  as your knowledge
% of TeX increases, you will probably want to edit this, but
% it should be fine as is for beginners.

% almost certainly you want these
\usepackage{amssymb}
\usepackage{amsmath}
\usepackage{amsfonts}

% used for TeXing text within eps files
%\usepackage{psfrag}
% need this for including graphics (\includegraphics)
%\usepackage{graphicx}
% for neatly defining theorems and propositions
%\usepackage{amsthm}
% making logically defined graphics
%%%\usepackage{xypic}

% there are many more packages, add them here as you need them

% define commands here
\begin{document}
Given a holomorphic function defined on some open set, one technique
for analytically continuing it to a larger set is by means of power
series.  One picks a point of the region and constructs the Taylor
series of the function about that point.  If it turns out that the
radius of convergence of the Taylor series is large enough that it
contains points which are not in the original domain, one can
extend the function to a larger domain obtrained by adding these points.
%%%%%
%%%%%
\end{document}
