\documentclass[12pt]{article}
\usepackage{pmmeta}
\pmcanonicalname{UnitDiskUpperHalfPlaneConformalEquivalenceTheorem}
\pmcreated{2013-03-22 13:37:52}
\pmmodified{2013-03-22 13:37:52}
\pmowner{CWoo}{3771}
\pmmodifier{CWoo}{3771}
\pmtitle{unit disk upper half plane conformal equivalence theorem}
\pmrecord{12}{34279}
\pmprivacy{1}
\pmauthor{CWoo}{3771}
\pmtype{Theorem}
\pmcomment{trigger rebuild}
\pmclassification{msc}{30C20}
\pmrelated{UnitDisk}
\pmrelated{UpperHalfPlane}
\pmrelated{MobiusTransformation}
\pmrelated{MobiusCircleTransformationTheorem}
\pmrelated{ConvertingBetweenThePoincareDiscModelAndTheUpperHalfPlaneModel}

\endmetadata

% this is the default PlanetMath preamble.  as your knowledge
% of TeX increases, you will probably want to edit this, but
% it should be fine as is for beginners.

% almost certainly you want these
\usepackage{amssymb}
\usepackage{amsmath}
\usepackage{amsfonts}

% used for TeXing text within eps files
%\usepackage{psfrag}
% need this for including graphics (\includegraphics)
%\usepackage{graphicx}
% for neatly defining theorems and propositions
\usepackage{amsthm}
% making logically defined graphics
%%%\usepackage{xypic}

% there are many more packages, add them here as you need them

% define commands here
\newtheorem{thm}{Theorem}
\begin{document}
\begin{thm} There is a conformal map from $\Delta$, the unit disk, to $UHP$, the upper half plane. \end{thm}

\begin{proof} Define $f \colon \hat{{\mathbb C}} \to \hat{{\mathbb C}}$ (where $\hat{{\mathbb C}}$ denotes the Riemann Sphere) to be $f(z) = \displaystyle{\frac{z-i}{z+i}}$. Notice that $f^{-1}(w)= \displaystyle{i \frac{1+w}{1-w}}$ and that $f$ (and therefore $f^{-1}$) is a Mobius transformation. 

Notice that $f(0)=-1$, $f(1)=\displaystyle{\frac{1-i}{1+i}} = -i$ and $f(-1) = \displaystyle{\frac{-1-i}{-1+i}} = i$. By the Mobius Circle Transformation Theorem, $f$ takes the real axis to the unit circle. Since $f(i)=0$, $f$ maps $UHP$ to $\Delta$ and $f^{-1}:\Delta \to UHP$. \end{proof}
%%%%%
%%%%%
\end{document}
