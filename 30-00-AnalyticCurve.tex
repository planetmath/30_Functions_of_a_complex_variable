\documentclass[12pt]{article}
\usepackage{pmmeta}
\pmcanonicalname{AnalyticCurve}
\pmcreated{2013-03-22 14:18:03}
\pmmodified{2013-03-22 14:18:03}
\pmowner{jirka}{4157}
\pmmodifier{jirka}{4157}
\pmtitle{analytic curve}
\pmrecord{7}{35759}
\pmprivacy{1}
\pmauthor{jirka}{4157}
\pmtype{Definition}
\pmcomment{trigger rebuild}
\pmclassification{msc}{30-00}
\pmclassification{msc}{54-00}
\pmsynonym{analytic arc}{AnalyticCurve}
\pmsynonym{smooth analytic curve}{AnalyticCurve}
\pmsynonym{real analytic curve}{AnalyticCurve}
\pmrelated{FreeAnalyticBoundaryArc}

\endmetadata

% this is the default PlanetMath preamble.  as your knowledge
% of TeX increases, you will probably want to edit this, but
% it should be fine as is for beginners.

% almost certainly you want these
\usepackage{amssymb}
\usepackage{amsmath}
\usepackage{amsfonts}

% used for TeXing text within eps files
%\usepackage{psfrag}
% need this for including graphics (\includegraphics)
%\usepackage{graphicx}
% for neatly defining theorems and propositions
\usepackage{amsthm}
% making logically defined graphics
%%%\usepackage{xypic}

% there are many more packages, add them here as you need them

% define commands here
\theoremstyle{theorem}
\newtheorem*{thm}{Theorem}
\newtheorem*{lemma}{Lemma}
\newtheorem*{conj}{Conjecture}
\newtheorem*{cor}{Corollary}
\newtheorem*{example}{Example}
\newtheorem*{prop}{Proposition}
\theoremstyle{definition}
\newtheorem*{defn}{Definition}
\theoremstyle{remark}
\newtheorem*{rmk}{Remark}

\begin{document}
There are several somewhat different definitions of the word \emph{analytic curve} depending on context.  
In the context of a real analytic manifold (for example ${\mathbb{R}}^n$), the most generic
definition is perhaps the following.

\begin{defn}
Suppose $X$ is a real analytic manifold.
A curve $\gamma \subset X$ is an \emph{analytic curve} if it is a real analytic submanifold
of dimension 1.  Equivalently if near each point $p \in \gamma,$
there exists a real analytic mapping
$f \colon (-1,1) \to X,$ such that $f$ has nonvanishing differential and maps onto a neighbourhood of $p$ in $\gamma .$ 
\end{defn}

It is sometimes common to equate the mapping $f$ and the curve $\gamma$.
If the curve is as above but instead in the complex plane, we can instead make the following equivalent definition.

\begin{defn}
A curve $\gamma \subset \mathbb{C}$ is said to be an {\em analytic curve} (or {\em analytic arc}) if every point of $\gamma$ has an open neighbourhood $\Delta$ for which there is an
onto conformal map $f \colon {\mathbb{D}} \to \Delta$ (where ${\mathbb{D}} \subset \mathbb{C}$ is
the unit disc) such that ${\mathbb{D}} \cap {\mathbb{R}}$ is mapped
onto $\Delta \cap \gamma$ by $f.$
\end{defn}

Other words for this concept are \emph{smooth analytic curve}, in which case the word \emph{analytic curve}
would be reserved for curves with singularities.  That is, for real analytic subvarieties of $X$.  Some authors will emphasize the fact that this is a real curve and say \emph{real analytic curve}.

In the context of subvarieties the following definition may be used.

\begin{defn}
An \emph{analytic curve} is a complex analytic subvariety of dimension 1 of a complex manifold.
\end{defn}

Note that locally all complex analytic subvarieties of dimension 1 in ${\mathbb{C}}^2$ can be parametrized by a the Puiseux parametrization theorem.  Perhaps that is why there is the confusion in using the term.

\begin{thebibliography}{9}
\bibitem{Gamelin:complex}
Theodore~B.\@ Gamelin.
{\em \PMlinkescapetext{Complex Analysis}}.
Springer-Verlag, New York, New York, 2001.
\bibitem{Whitney:varieties}
Hassler Whitney.
{\em \PMlinkescapetext{Complex Analytic Varieties}}.
Addison-Wesley, Philippines, 1972.
\end{thebibliography}
%%%%%
%%%%%
\end{document}
