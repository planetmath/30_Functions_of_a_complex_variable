\documentclass[12pt]{article}
\usepackage{pmmeta}
\pmcanonicalname{ProofOfTheCauchyRiemannEquations}
\pmcreated{2013-03-22 12:55:39}
\pmmodified{2013-03-22 12:55:39}
\pmowner{rmilson}{146}
\pmmodifier{rmilson}{146}
\pmtitle{proof of the Cauchy-Riemann equations}
\pmrecord{6}{33282}
\pmprivacy{1}
\pmauthor{rmilson}{146}
\pmtype{Proof}
\pmcomment{trigger rebuild}
\pmclassification{msc}{30E99}
\pmdefines{complex derivative}

\endmetadata

\usepackage{amsmath}
\usepackage{amsfonts}
\usepackage{amssymb}
\newcommand{\reals}{\mathbb{R}}
\newcommand{\natnums}{\mathbb{N}}
\newcommand{\cnums}{\mathbb{C}}
\newcommand{\znums}{\mathbb{Z}}
\newcommand{\lp}{\left(}
\newcommand{\rp}{\right)}
\newcommand{\lb}{\left[}
\newcommand{\rb}{\right]}
\newcommand{\supth}{^{\text{th}}}
\newtheorem{proposition}{Proposition}
\newtheorem{definition}[proposition]{Definition}
\newcommand{\nl}[1]{{\PMlinkescapetext{{#1}}}}
\newcommand{\pln}[2]{{\PMlinkname{{#1}}{#2}}}
\begin{document}
\paragraph{Existence of complex derivative implies the Cauchy-Riemann
  equations.}

Suppose that the complex
derivative
\begin{equation}
  \label{eq:cder}
f'(z) = \lim_{\zeta\rightarrow 0} \frac{f(z+\zeta)-f(z)}{\zeta}  
\end{equation}
exists for some $z\in \cnums$. 
This means that for all $\epsilon>0$, there exists a $\rho>0$, such that
for all complex $\zeta$ with
$\vert \zeta\vert<\rho$, we have
$$\left\vert f'(z)  - \frac{f(z+\zeta)-f(z)}{\zeta} \right \vert<\epsilon.$$

Henceforth, set
$$f=u+iv,\quad z=x+iy.$$
If $\zeta$ is
real, then the above limit reduces to a partial derivative in $x$, i.e.
$$f'(z) = \frac{\partial f}{\partial x} = \frac{\partial u}{\partial
  x} + i \frac{\partial v}{\partial x},$$
Taking the limit with an
imaginary $\zeta$ we deduce that
$$f'(z) = -i\frac{\partial f}{\partial y} = -i \frac{\partial u}{\partial
  y} + \frac{\partial v}{\partial y}.$$
Therefore
$$\frac{\partial f}{\partial x} = -i\frac{\partial f}{\partial y},$$
and breaking this relation up into its real and imaginary parts gives
the Cauchy-Riemann equations.

\paragraph{The Cauchy-Riemann
  equations imply the existence of a complex derivative.}
Suppose that the Cauchy-Riemann
equations 
$$
\frac{\partial u}{\partial x} = \frac{\partial v}{\partial y},\quad
\frac{\partial u}{\partial y} = -\frac{\partial v}{\partial x},
$$
hold for a fixed $(x,y)\in\reals^2$, 
and  that all the
partial derivatives are continuous at $(x,y)$ as well.  The continuity
implies that all directional derivatives exist as well.  In
other words, for $\xi,\eta\in\reals$ and $\rho=\sqrt{\xi^2+\eta^2}$
we have
$$
\frac{
u(x+\xi,y+\eta) - u(x,y) - (\xi \frac{\partial u}{\partial x } + \eta
\frac{\partial u}{\partial y})}{\rho} \rightarrow
0,\;\mbox{as } \rho\rightarrow 0,$$
with a similar relation holding for $v(x,y)$.  Combining the two scalar
relations into a vector relation we obtain
$$
\rho^{-1} \left\Vert
  \begin{pmatrix}
    u(x+\xi,y+\eta) \\ v(x+\xi,y+\eta)
  \end{pmatrix}
  -
  \begin{pmatrix}
    u(x,y) \\ v(x,y)
  \end{pmatrix}
  -
  \begin{pmatrix}
    \frac{\partial u}{\partial x} & \frac{\partial u}{\partial y}\\
    \frac{\partial v}{\partial x} & \frac{\partial v}{\partial y}\\
  \end{pmatrix}
  \begin{pmatrix}
    \xi \\ \eta
  \end{pmatrix}
\right\Vert \rightarrow 0,\;\mbox{as } \rho\rightarrow 0.$$
Note that
the Cauchy-Riemann equations imply that the matrix-vector product
above is equivalent to the product of two complex numbers, namely
$$\left(\frac{\partial u}{\partial x} +i\frac{\partial v}{\partial
  x}\right)(\xi+i\eta).$$
Setting
\begin{eqnarray*}
f(z) &=& u(x,y)+i v(x,y),\\
f'(z) &=& \frac{\partial u}{\partial x} + i\frac{\partial v}{\partial
  x}\\
\zeta &=& \xi+i\eta  
\end{eqnarray*}
we can therefore rewrite the above limit relation  as
$$
\left\vert \frac{ f(z+\zeta)-f(z) - f'(z)\zeta}{ \zeta}\right\vert  \rightarrow
0,\;\mbox{as } \rho\rightarrow 0,$$
which is the complex limit definition of $f'(z)$ shown
in \eqref{eq:cder}.
%%%%%
%%%%%
\end{document}
