\documentclass[12pt]{article}
\usepackage{pmmeta}
\pmcanonicalname{Holomorphic}
\pmcreated{2013-03-22 12:04:33}
\pmmodified{2013-03-22 12:04:33}
\pmowner{djao}{24}
\pmmodifier{djao}{24}
\pmtitle{holomorphic}
\pmrecord{12}{31146}
\pmprivacy{1}
\pmauthor{djao}{24}
\pmtype{Definition}
\pmcomment{trigger rebuild}
\pmclassification{msc}{30D20}
\pmclassification{msc}{32A10}
\pmsynonym{holomorphic function}{Holomorphic}
\pmsynonym{regular function}{Holomorphic}
\pmsynonym{complex differentiable}{Holomorphic}
\pmrelated{CauchyRiemannEquations}
\pmrelated{Analytic}

\usepackage{amssymb}
\usepackage{amsmath}
\usepackage{amsfonts}
\usepackage{graphicx}
%%%\usepackage{xypic}
\begin{document}
Let $U \subset \mathbb{C}$ be a domain in the complex numbers. A
function $f \colon U \longrightarrow \mathbb{C}$ is \emph{holomorphic} if $f$
has a complex derivative at every point $x$ in $U$, i.e. if 
$$\lim_{z\rightarrow z_0} \frac{f(z)-f(z_0)}{z-z_0}$$
exists for all $z_0\in U$.

More generally, if $\Omega\subset \mathbb{C}^n$ is a domain, then a function $f\colon \Omega \to \mathbb{C}$ is said to be \emph{holomorphic} if $f$ is holomorphic in each of the variables. The class of all holomorphic functions on $\Omega$ is usually denoted by $\mathcal{O}(\Omega)$.
%%%%%
%%%%%
%%%%%
\end{document}
