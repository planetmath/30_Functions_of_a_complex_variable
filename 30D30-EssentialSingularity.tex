\documentclass[12pt]{article}
\usepackage{pmmeta}
\pmcanonicalname{EssentialSingularity}
\pmcreated{2013-03-22 13:32:10}
\pmmodified{2013-03-22 13:32:10}
\pmowner{pbruin}{1001}
\pmmodifier{pbruin}{1001}
\pmtitle{essential singularity}
\pmrecord{7}{34132}
\pmprivacy{1}
\pmauthor{pbruin}{1001}
\pmtype{Definition}
\pmcomment{trigger rebuild}
\pmclassification{msc}{30D30}
\pmrelated{LaurentSeries}
\pmrelated{Pole}
\pmrelated{RemovableSingularity}
\pmrelated{PicardsTheorem}
\pmrelated{RiemannsRemovableSingularityTheorem}

\endmetadata

% this is the default PlanetMath preamble.  as your knowledge
% of TeX increases, you will probably want to edit this, but
% it should be fine as is for beginners.

% almost certainly you want these
\usepackage{amssymb}
\usepackage{amsmath}
\usepackage{amsfonts}

% used for TeXing text within eps files
%\usepackage{psfrag}
% need this for including graphics (\includegraphics)
%\usepackage{graphicx}
% for neatly defining theorems and propositions
%\usepackage{amsthm}
% making logically defined graphics
%%%\usepackage{xypic}

% there are many more packages, add them here as you need them

% define commands here
\begin{document}
Let $U\subset\mathbb{C}$ be a domain, $a\in U$, and let $f:U \setminus \{a\} \to \mathbb{C}$ be holomorphic.  If the Laurent series expansion of $f(z)$ around $a$ contains infinitely many terms with negative powers of $z-a$, then $a$ is said to be an \emph{essential singularity} of $f$.  Any singularity of $f$ is a removable singularity, a pole or an essential singularity.

If $a$ is an essential singularity of $f$, then the image of any punctured neighborhood of $a$ under $f$ is dense in $\mathbb{C}$ (the Casorati-Weierstrass theorem).  In fact, an even stronger statement is true: according to Picard's theorem, the image of any punctured neighborhood of $a$ is $\mathbb{C}$, with the possible exception of a single point.
%%%%%
%%%%%
\end{document}
