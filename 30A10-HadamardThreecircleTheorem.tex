\documentclass[12pt]{article}
\usepackage{pmmeta}
\pmcanonicalname{HadamardThreecircleTheorem}
\pmcreated{2013-03-22 14:10:45}
\pmmodified{2013-03-22 14:10:45}
\pmowner{bbukh}{348}
\pmmodifier{bbukh}{348}
\pmtitle{Hadamard three-circle theorem}
\pmrecord{7}{35605}
\pmprivacy{1}
\pmauthor{bbukh}{348}
\pmtype{Theorem}
\pmcomment{trigger rebuild}
\pmclassification{msc}{30A10}
\pmclassification{msc}{30C80}
\pmrelated{MaximumPrinciple}
\pmrelated{LogarithmicallyConvexFunction}
\pmrelated{HardysTheorem}

\endmetadata

\usepackage{amssymb}
\usepackage{amsmath}
\usepackage{amsfonts}

\newcommand*{\abs}[1]{\left\lvert #1\right\rvert}

\makeatletter
\@ifundefined{bibname}{}{\renewcommand{\bibname}{References}}
\makeatother
\begin{document}
Let $f(z)$ be a complex analytic function on the annulus
$r_1\leq\abs{z}\leq r_3$. Let $M(r)$ be the maximum of
$\abs{f(z)}$ on the circle $\abs{z}=r$. Then $\log M(r)$ is a
convex function of $\log r$. Moreover, if $f(z)$ is not of the form $cz^\lambda$ for some $\lambda$, then $\log M(r)$ is a \PMlinkname{strictly convex}{ConvexFunction} as a function of $\log r$.

The conclusion of the theorem can be restated as
\begin{equation*}
\log\frac{r_3}{r_1} \log M(r_2) \leq \log\frac {r_3}{r_2} \log M(r_1) + 
\log\frac {r_2}{r_1} \log M(r_3)
\end{equation*}
for any three concentric circles of radii $r_1<r_2<r_3$.
%%%%%
%%%%%
\end{document}
