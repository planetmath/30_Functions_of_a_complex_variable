\documentclass[12pt]{article}
\usepackage{pmmeta}
\pmcanonicalname{ZeroesOfAnalyticFunctionsAreIsolated}
\pmcreated{2013-03-22 13:38:10}
\pmmodified{2013-03-22 13:38:10}
\pmowner{brianbirgen}{2180}
\pmmodifier{brianbirgen}{2180}
\pmtitle{zeroes of analytic functions are isolated}
\pmrecord{8}{34285}
\pmprivacy{1}
\pmauthor{brianbirgen}{2180}
\pmtype{Result}
\pmcomment{trigger rebuild}
\pmclassification{msc}{30C15}
\pmsynonym{zeros of analytic functions are isolated}{ZeroesOfAnalyticFunctionsAreIsolated}
\pmrelated{Complex}
\pmrelated{LeastAndGreatestZero}
\pmrelated{IdentityTheorem}
\pmrelated{WhenAllSingularitiesArePoles}

\endmetadata

% this is the default PlanetMath preamble.  as your knowledge
% of TeX increases, you will probably want to edit this, but
% it should be fine as is for beginners.

% almost certainly you want these
\usepackage{amssymb}
\usepackage{amsmath}
\usepackage{amsfonts}

% used for TeXing text within eps files
%\usepackage{psfrag}
% need this for including graphics (\includegraphics)
%\usepackage{graphicx}
% for neatly defining theorems and propositions
%\usepackage{amsthm}
% making logically defined graphics
%%%\usepackage{xypic}

% there are many more packages, add them here as you need them

% define commands here
\begin{document}
The zeroes of a non-constant analytic function on ${\mathbb C}$ are isolated. 
Let $f$ be an analytic function 
defined in some domain $D \subset {\mathbb C}$ and 
let $f(z_0)=0$ for some $z_0 \in D$. Because $f$ is analytic, 
there is a Taylor series expansion for $f$ around $z_0$ which 
converges on an open disk $|z-z_0|<R$. Write it as 
$f(z) = \Sigma_{n=k}^{\infty} a_n (z-z_0)^n$, with $a_k \ne 0$ and $k > 0$ 
($a_k$ is the first non-zero term). 
One can factor the series so that 
$f(z) = (z-z_0)^k \Sigma_{n=0}^{\infty} a_{n+k} (z-z_0)^n$ and define 
$g(z) = \Sigma_{n=0}^{\infty} a_{n+k} (z-z_0)^n$ so that $f(z) = (z-z_0)^k g(z)$. 
Observe that $g(z)$ is analytic on $|z-z_0|<R$. 

To show that $z_0$ is an isolated zero of $f$, 
we must find $\epsilon > 0$ so that $f$ is non-zero on $0<|z-z_0|<\epsilon$. 
It is enough to find $\epsilon>0$ so that $g$ is non-zero 
on $|z-z_0|<\epsilon$ by the relation $f(z) = (z-z_0)^k g(z)$. 
Because $g(z)$ is analytic, it is continuous at $z_0$. 
Notice that $g(z_0)=a_k \ne 0$, 
so there exists an $\epsilon > 0$ so that for all $z$ with
$|z-z_0| < \epsilon$ it follows that $|g(z) - a_k| < \frac{|a_k|}{2}$. 
This implies that $g(z)$ is non-zero in this set.
%%%%%
%%%%%
\end{document}
