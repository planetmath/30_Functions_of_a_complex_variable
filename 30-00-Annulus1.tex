\documentclass[12pt]{article}
\usepackage{pmmeta}
\pmcanonicalname{Annulus1}
\pmcreated{2013-03-22 13:34:52}
\pmmodified{2013-03-22 13:34:52}
\pmowner{Wkbj79}{1863}
\pmmodifier{Wkbj79}{1863}
\pmtitle{annulus}
\pmrecord{7}{34202}
\pmprivacy{1}
\pmauthor{Wkbj79}{1863}
\pmtype{Definition}
\pmcomment{trigger rebuild}
\pmclassification{msc}{30-00}
\pmsynonym{open annulus}{Annulus1}
\pmsynonym{annular region}{Annulus1}
\pmrelated{Annulus}
\pmdefines{closed annulus}

\endmetadata

% this is the default PlanetMath preamble.  as your knowledge
% of TeX increases, you will probably want to edit this, but
% it should be fine as is for beginners.

% almost certainly you want these
\usepackage{amssymb}
\usepackage{amsmath}
\usepackage{amsfonts}

% used for TeXing text within eps files
%\usepackage{psfrag}
% need this for including graphics (\includegraphics)
%\usepackage{graphicx}
% for neatly defining theorems and propositions
%\usepackage{amsthm}
% making logically defined graphics
%%%\usepackage{xypic}

% there are many more packages, add them here as you need them

% define commands here
\newcommand{\bbC}{\mathbb{C}}
\newcommand{\bbF}{\mathbb{F}}
\newcommand{\bbN}{\mathbb{N}}
\newcommand{\bbR}{\mathbb{R}}
\newcommand{\bbZ}{\mathbb{Z}}
\newcommand{\calO}{\mathcal{O}}
\newcommand{\fkm}{\mathfrak{m}}
\newcommand{\fkp}{\mathfrak{p}}
\newcommand{\Gal}{\mathrm{Gal}}
\newcommand{\Hom}{\mathrm{Hom}}
\newcommand{\ov}[1]{\overline{#1}}
\newcommand{\ur}{\mathrm{ur}}
\newcommand{\vp}{\varphi}
\newcommand{\wh}[1]{\widehat{#1}}
\newcommand{\wt}[1]{\widetilde{#1}}
\newcommand{\smashedleftarrow}{\setbox0=\hbox{$\longleftarrow$}\ht0=1pt\box0}
\newcommand{\smashedrightarrow}{\setbox0=\hbox{$\longrightarrow$}\ht0=1pt\box0}
\newcommand{\lisom}{\buildrel{\hskip+0.04cm\sim}\over{\smashedleftarrow}}
\newcommand{\risom}{\buildrel{\hskip-0.04cm\sim}\over{\smashedrightarrow}}
\begin{document}
An \emph{annulus} is the region bounded between two (usually concentric) circles.

An \emph{open annulus} is a domain in the complex plane of the form
\[
A = A_w(r,R) = \{z \in \bbC : r < |z-w| < R\},
\]
where $w$ is an arbitrary complex number, and $r$ and $R$ are real numbers with $0 < r < R$.  Such a set is often called an \emph{annular region}.

It should be noted that the \PMlinkescapetext{word} annulus usually refers to an open annulus.

More generally, one can allow $r = 0$ or $R = \infty$.  (This makes sense for the purposes of the bound on $|z-w|$ above.)  This would make an annulus include the cases of a punctured disc, and some unbounded domains.

Analogously, a \emph{closed annulus} is a set of the form
\[
\ov{A} = \ov{A}_w(r,R) = \{z \in \bbC : r \leq |z-w| \leq R\},
\]
where $w \in \bbC$, and $r$ and $R$ are real numbers with $0 < r < R$.

One can show that two annuli $D_w(r,R)$ and $D_{w'}(r',R')$ are conformally equivalent if and only if $R/r = R'/r'$.  More generally, the complement of any closed disk in an open disk is conformally equivalent to precisely one annulus of the form $D_0(r,1)$.
%%%%%
%%%%%
\end{document}
