\documentclass[12pt]{article}
\usepackage{pmmeta}
\pmcanonicalname{TrigonometricFormulasFromDeMoivreIdentity}
\pmcreated{2013-03-22 18:51:16}
\pmmodified{2013-03-22 18:51:16}
\pmowner{pahio}{2872}
\pmmodifier{pahio}{2872}
\pmtitle{trigonometric formulas from de Moivre identity}
\pmrecord{16}{41664}
\pmprivacy{1}
\pmauthor{pahio}{2872}
\pmtype{Derivation}
\pmcomment{trigger rebuild}
\pmclassification{msc}{30D05}
\pmclassification{msc}{30A99}
\pmrelated{TrigonometricFormulasFromSeries}
\pmrelated{ReductionFormulas}
\pmrelated{GoniometricFormulae}

% this is the default PlanetMath preamble.  as your knowledge
% of TeX increases, you will probably want to edit this, but
% it should be fine as is for beginners.

% almost certainly you want these
\usepackage{amssymb}
\usepackage{amsmath}
\usepackage{amsfonts}

% used for TeXing text within eps files
%\usepackage{psfrag}
% need this for including graphics (\includegraphics)
%\usepackage{graphicx}
% for neatly defining theorems and propositions
 \usepackage{amsthm}
% making logically defined graphics
%%%\usepackage{xypic}

% there are many more packages, add them here as you need them

% define commands here

\theoremstyle{definition}
\newtheorem*{thmplain}{Theorem}

\begin{document}
\PMlinkescapeword{terms}
De Moivre identity
\begin{align}
(\cos\varphi+i\sin\varphi)^n \,=\, \cos{n\varphi}+i\sin{n\varphi} \qquad (n \in \mathbb{Z})
\end{align}
implies simply some important trigonometric formulas, the derivation of which without imaginary numbers would require much longer calculations.\\


When one expands the left hand side of (1) using the binomial theorem ($n > 0$), the sum of the real terms (the real part) must be $\cos{n\varphi}$ and the sum of the imaginary terms (cf. the imaginary part) must equal $i\sin{n\varphi}$.\, Thus both $\cos{n\varphi}$ and 
$\sin{n\varphi}$ has been expressed as polynomials of $\sin\varphi$ and $\cos\varphi$ with integer coefficients.\\

For example, if\, $n = 5$,\, we have
$$(\cos\varphi+i\sin\varphi)^5 \;=\;
\cos^5\varphi+5i\cos^4\varphi\sin\varphi-10\cos^3\varphi\sin^2\varphi
-10i\cos^2\varphi\sin^3\varphi+5\cos\varphi\sin^4\varphi+i\sin^5\varphi,$$
whence
$$\cos{5\varphi} \;=\; \cos^5\varphi-10\cos^3\varphi\sin^2\varphi+5\cos\varphi\sin^4\varphi,$$
$$\sin{5\varphi} \;=\; 5\cos^4\varphi\sin\varphi-10\cos^2\varphi\sin^3\varphi+\sin^5\varphi.$$
By the ``fundamental formula''\, $\sin^2\varphi+\cos^2\varphi = 1$\, of trigonometry, the even powers on the right hand sides may be expressed with the other function; therefore we obtain
\begin{align}
\cos{5\varphi} \;=\; 16\cos^5\varphi-20\cos^3\varphi+5\cos\varphi,
\end{align}
\begin{align}
\sin{5\varphi} \;=\; 16\sin^5\varphi-20\sin^3\varphi+5\sin\varphi.
\end{align}

\subsection{Linearisation formulas}
There are also inverse formulas where one expresses the integer powers $\cos^m\varphi$ and $\sin^n\varphi$ and their products as the polynomials with rational coefficients of either $\cos\varphi$, $\cos2\varphi$, \ldots\, or 
$\sin\varphi$, $\sin2\varphi$, \ldots,\, depending on whether it is a question of an \PMlinkname{even}{EvenFunction} or an odd function of $\varphi$.\, We will derive the transformation formulas.

If we denote
$$\cos\varphi+i\sin\varphi \;:=\; t,$$
then the complex conjugate of $t$ is the same as its inverse number:
$$\cos\varphi-i\sin\varphi \;=\; \frac{1}{t}.$$
By adding and subtracting, these equations yield
\begin{align}
\cos\varphi \;=\; \frac{1}{2}\left(t+\frac{1}{t}\right), \quad \sin\varphi \;=\; \frac{1}{2i}\left(t-\frac{1}{t}\right).
\end{align}
Similarly, the equations
$$(\cos\varphi+i\sin\varphi)^{\pm n} \,=\, \cos(\pm n\varphi)+i\sin(\pm n\varphi)$$
yield
\begin{align}
\cos{n\varphi} \;=\; \frac{1}{2}\left(t^n+\frac{1}{t^n}\right), \quad 
\sin{n\varphi} \;=\; \frac{1}{2i}\left(t^n-\frac{1}{t^n}\right).
\end{align}
for any integer $n$.\, The linearisation formulas are obtained by expanding first the expression to be linearised with the equations (4) and then simplifying the result with the equations (5).\\

\textbf{Example 1.}
\begin{align*}
\cos^4\varphi & \;=\; \left(\frac{1}{2}\left(t+\frac{1}{t}\right)\right)^4\\
& \;=\; \frac{1}{16}\left(t^4+4t^2+6+\frac{4}{t^2}+\frac{1}{t^4}\right)\\
& \;=\; \frac{1}{16}\left(t^4+\frac{1}{t^4}\right)+\frac{1}{4}\left(t^2+\frac{1}{t^2}\right)+\frac{3}{8}\\
& \;=\; \frac{1}{8}\cos4\varphi+\frac{1}{2}\cos2\varphi+\frac{3}{8}
\end{align*}

\textbf{Example 2.}
\begin{align*}
\cos^4\varphi\sin^3\varphi &\;=\; \frac{1}{16}\left(t+\frac{1}{t}\right)^4\frac{-1}{8i}\left(t-\frac{1}{t}\right)^3\\
&\;=\; -\frac{1}{128i}\left(t^2-\frac{1}{t^2}\right)^3\left(t+\frac{1}{t}\right)\\
&\;=\; -\frac{1}{128i}\left(t^6-3t^2+\frac{3}{t^2}-\frac{1}{t^6}\right)\left(t+\frac{1}{t}\right)\\
&\;=\; -\frac{1}{128i}\left(t^7-3t^3+\frac{3}{t}-\frac{1}{t^5}-3t+\frac{3}{t^3}-\frac{1}{t^7}\right)\\
&\;=\; -\frac{1}{128i}\left(\left(t^7-\frac{1}{t^7}\right)+\left(t^5-\frac{1}{t^5}\right)
-3\left(t^3-\frac{1}{t^3}\right)-3\left(t-\frac{1}{t}\right)\right)\\
&\;=\; -\frac{1}{64}\sin7\varphi-\frac{1}{64}\sin5\varphi+\frac{3}{64}\sin3\varphi+\frac{3}{64}\sin\varphi
\end{align*}
%%%%%
%%%%%
\end{document}
