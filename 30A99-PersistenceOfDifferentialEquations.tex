\documentclass[12pt]{article}
\usepackage{pmmeta}
\pmcanonicalname{PersistenceOfDifferentialEquations}
\pmcreated{2013-03-22 16:20:35}
\pmmodified{2013-03-22 16:20:35}
\pmowner{rspuzio}{6075}
\pmmodifier{rspuzio}{6075}
\pmtitle{persistence of differential equations}
\pmrecord{10}{38475}
\pmprivacy{1}
\pmauthor{rspuzio}{6075}
\pmtype{Corollary}
\pmcomment{trigger rebuild}
\pmclassification{msc}{30A99}

\endmetadata

% this is the default PlanetMath preamble.  as your knowledge
% of TeX increases, you will probably want to edit this, but
% it should be fine as is for beginners.

% almost certainly you want these
\usepackage{amssymb}
\usepackage{amsmath}
\usepackage{amsfonts}

% used for TeXing text within eps files
%\usepackage{psfrag}
% need this for including graphics (\includegraphics)
%\usepackage{graphicx}
% for neatly defining theorems and propositions
%\usepackage{amsthm}
% making logically defined graphics
%%%\usepackage{xypic}

% there are many more packages, add them here as you need them

% define commands here

\begin{document}
The persistence of analytic relations has important consequences 
for the theory of differential equations in the complex plane.
Suppose that a function $f$ satisfies a differential equation
$F(x,f(x), f'(x), \ldots, f^{(n)}) = 0$ where $F$ is a polynomial.  
This equation may be viewed as a polynomial relation between the 
$n+2$ functions $\mathrm{id},f, f', \ldots, f^{(n)}$ hence, by the 
persistence of analytic relations, it will also hold for the analytic
continuations of these functions.  In other words, if an
algebraic differential equation holds for a function in some
region, it will still hold when that function is analytically 
continued to a larger region.

An interesting special case is that of the homogeneous linear 
differential equation with polynomial coefficients.  In that
case, we have the principle of superposition which guarantees 
that a linear combination of solutions is also a solution. 
Hence, if we start with a basis of solutions to our equation 
about some point and analytically continue them back to our 
starting point, we obtain linear combinations of those 
solutions.  This observation plays a very important role in the
theory of differential equations in the complex plane and is
the foundation for the notion of monodromy group and Riemann's 
global characterization of the hypergeometric function.  

For a less exalted illustrative example, we can consider the
complex logarithm.  The differential equation
 \[x y'' + y' = 0\]
has as solutions $y = 1$ and $y = \log x$.  While the former 
is as singly valued as functions get, the latter is multiply
valued.  Hence upon performong analytic continuation, we expect
that the second solution will continue to a linear combination
of the two solutions.  This, of course is exactly what happens;
upon analytic continuation, the second solution becomes the
solution $y = \log x + n \pi i$ where $n$ is an integer whose
value depends on how we carry out the analytic continuation. 
%%%%%
%%%%%
\end{document}
