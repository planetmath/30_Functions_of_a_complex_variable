\documentclass[12pt]{article}
\usepackage{pmmeta}
\pmcanonicalname{HarmonicConjugateFunction}
\pmcreated{2013-03-22 14:45:11}
\pmmodified{2013-03-22 14:45:11}
\pmowner{pahio}{2872}
\pmmodifier{pahio}{2872}
\pmtitle{harmonic conjugate function}
\pmrecord{21}{36392}
\pmprivacy{1}
\pmauthor{pahio}{2872}
\pmtype{Definition}
\pmcomment{trigger rebuild}
\pmclassification{msc}{30F15}
\pmclassification{msc}{31A05}
\pmsynonym{harmonic conjugate}{HarmonicConjugateFunction}
\pmsynonym{conjugate harmonic function}{HarmonicConjugateFunction}
\pmsynonym{conjugate harmonic}{HarmonicConjugateFunction}
\pmrelated{ComplexConjugate}
\pmrelated{OrthogonalCurves}
\pmrelated{TopicEntryOnComplexAnalysis}
\pmrelated{ExactDifferentialEquation}

% this is the default PlanetMath preamble.  as your knowledge
% of TeX increases, you will probably want to edit this, but
% it should be fine as is for beginners.

% almost certainly you want these
\usepackage{amssymb}
\usepackage{amsmath}
\usepackage{amsfonts}

% used for TeXing text within eps files
%\usepackage{psfrag}
% need this for including graphics (\includegraphics)
%\usepackage{graphicx}
% for neatly defining theorems and propositions
%\usepackage{amsthm}
% making logically defined graphics
%%%\usepackage{xypic}

% there are many more packages, add them here as you need them

% define commands here
\begin{document}
Two harmonic functions $u$ and $v$ from an \PMlinkname{open}{OpenSet} subset $A$ of $\mathbb{R}\!\times\!\mathbb{R}$ to $\mathbb{R}$, which satisfy the Cauchy-Riemann equations
\begin{align}
u_x \;=\; v_y, \quad u_y \;=\; -v_x,
\end{align}
are the {\em harmonic conjugate functions} of each other. 

\begin{itemize}

\item The relationship between $u$ and $v$ has a \PMlinkescapetext{simple} geometric meaning:\, Let's determine the slopes of the constant-value curves\, $u(x,\,y) = a$\, and\, $v(x,\,y) = b$\, in any point\, $(x,\,y)$\, by differentiating these equations.\, The first gives\, $u_x dx+u_y dy = 0$,\, or
 $$\frac{dy}{dx}^{(u)} \;=\; -\frac{u_x}{u_y} \;=\; \tan\alpha,$$
and the second similarly
 $$\frac{dy}{dx}^{(v)} \;=\; -\frac{v_x}{v_y}$$
but this is, by virtue of (1), equal to
 $$\frac{u_y}{u_x} \;=\; -\frac{1}{\tan\alpha}.$$
Thus, by the condition of orthogonality, the curves intersect at right angles in every point.

\item If one of $u$ and $v$ is known, then the other may be determined with (1):\, When e.g. the function $u$ is known, we need only to \PMlinkescapetext{calculate} the line integral
$$v(x, y) \;=\; \int_{(x_0, y_0)}^{(x, y)}(-u_y\,dx+u_x\,dy)$$
along any \PMlinkescapetext{path} connecting \,$(x_0,\,y_0)$\, and\, 
$(x,\,y)$\, in $A$.\, The result is the harmonic conjugate $v$ of $u$, unique up to a real addend if $A$ is simply connected.

\item It follows from the preceding, that every harmonic function has a harmonic conjugate function.

\item The real part and the imaginary part of a holomorphic function are always the harmonic conjugate functions of each other.

\end{itemize}


\textbf{Example.}\, $\sin{x}\cosh{y}$\, and\, $\cos{x}\sinh{y}$\, are harmonic conjugates of each other.

%%%%%
%%%%%
\end{document}
