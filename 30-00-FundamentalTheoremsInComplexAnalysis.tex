\documentclass[12pt]{article}
\usepackage{pmmeta}
\pmcanonicalname{FundamentalTheoremsInComplexAnalysis}
\pmcreated{2013-03-22 14:57:33}
\pmmodified{2013-03-22 14:57:33}
\pmowner{rspuzio}{6075}
\pmmodifier{rspuzio}{6075}
\pmtitle{fundamental theorems in complex analysis}
\pmrecord{26}{36656}
\pmprivacy{1}
\pmauthor{rspuzio}{6075}
\pmtype{Topic}
\pmcomment{trigger rebuild}
\pmclassification{msc}{30-00}
\pmrelated{TopicEntryOnComplexAnalysis}

% this is the default PlanetMath preamble.  as your knowledge
% of TeX increases, you will probably want to edit this, but
% it should be fine as is for beginners.

% almost certainly you want these
\usepackage{amssymb}
\usepackage{amsmath}
\usepackage{amsfonts}

% used for TeXing text within eps files
%\usepackage{psfrag}
% need this for including graphics (\includegraphics)
%\usepackage{graphicx}
% for neatly defining theorems and propositions
%\usepackage{amsthm}
% making logically defined graphics
%%%\usepackage{xypic}

% there are many more packages, add them here as you need them

% define commands here
\begin{document}
The following is a list of fundamental theorems in the subject of complex analysis (single complex variable).  If a theorem does not yet appear in the encyclopedia, please consider adding it --- Planet Math is a work in progress and \PMlinkescapetext{even} some basic results have not yet been entered.  Likewise, if some basic theorem has been overlooked in this list, please add it.

\begin{itemize}
\item Cauchy-Riemann equations
\item Cauchy's integral theorem
\item second form of Cauchy integral theorem
\item Morera's theorem
\item Cauchy's integral formula
\item Cauchy's residue theorem
\item Cauchy's argument principle
\item Rouch\'e's theorem
\item identity theorem of power series
\item rigidity theorem for analytic functions
\item Riemann's removable singularity theorem 
\item Casorati-Weierstrass theorem
\item implicit function theorem for complex analytic functions (I gave proofs of this and the next theorem in a posting to a forum and must convert them to an encyclopaedia entry.)
\item inverse function theorem for complex analytic functions
\item maximal modulus principle
\item Schwarz lemma
\item Liouville's theorem
\item characterization of rational functions
\item Weierstrass' factorization theorem
\item Weierstrass' criterion of uniform convergence
\item Mittag-Leffler's theorem 
\item Möbius circle transformation theorem 
\item Riemann mapping theorem
\item Gauss' mean value theorem
\item Schwarz' reflection principle
\item Harnack's principle
\item Bloch theorem
\item \PMlinkname{Picard's theorem}{PicardsTheorem}
\item Little Picard theorem
\item Monodromy theorem
\item Runge's theorem
\item Mergelyan's theorem
\item Montel's theorem
\item Marty's theorem
\item Hurwitz's theorem
\item Bieberbach's conjecture
\item Koebe one-fourth theorem
\item \PMlinkname{Factorization theorem for $H^\infty$ functions}{FactorizationTheoremForHinftyFunctions}
\item Plemelj formulas
\item Harnack theorem
\item Schwarz and Poisson formulas 
\end{itemize}
%%%%%
%%%%%
\end{document}
