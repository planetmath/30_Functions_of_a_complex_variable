\documentclass[12pt]{article}
\usepackage{pmmeta}
\pmcanonicalname{TheoremsOnComplexFunctionSeries}
\pmcreated{2013-03-22 16:47:55}
\pmmodified{2013-03-22 16:47:55}
\pmowner{pahio}{2872}
\pmmodifier{pahio}{2872}
\pmtitle{theorems on complex function series}
\pmrecord{9}{39032}
\pmprivacy{1}
\pmauthor{pahio}{2872}
\pmtype{Theorem}
\pmcomment{trigger rebuild}
\pmclassification{msc}{30B99}
\pmclassification{msc}{40A30}
\pmrelated{IdentityTheoremOfPowerSeries}
\pmrelated{WeierstrassDoubleSeriesTheorem}

% this is the default PlanetMath preamble.  as your knowledge
% of TeX increases, you will probably want to edit this, but
% it should be fine as is for beginners.

% almost certainly you want these
\usepackage{amssymb}
\usepackage{amsmath}
\usepackage{amsfonts}

% used for TeXing text within eps files
%\usepackage{psfrag}
% need this for including graphics (\includegraphics)
%\usepackage{graphicx}
% for neatly defining theorems and propositions
 \usepackage{amsthm}
% making logically defined graphics
%%%\usepackage{xypic}

% there are many more packages, add them here as you need them

% define commands here

\theoremstyle{definition}
\newtheorem*{thmplain}{Theorem}

\begin{document}
\textbf{Theorem 1.}\, If the complex functions\, $f_1,\,f_2,\,f_3,\,\ldots$\, are continuous on the path $\gamma$ and the series 
\begin{align}
f_1(z)+f_2(z)+f_3(z)+\ldots
\end{align}
converges uniformly on $\gamma$ to the sum function $F$, then one has
$$\int_{\gamma}F(z)\,dz \;=\; \int_{\gamma}f_1(z)\,dz+\int_{\gamma}f_2(z)\,dz+\int_{\gamma}f_3(z)\,dz+\ldots$$

\textbf{Theorem 2.}\, If the functions\, $f_1,\,f_2,\,f_3,\,\ldots$\, are holomorphic in a domain $A$ and the series (1) converges uniformly in every \PMlinkname{closed}{ClosedSet} disc of $A$, then also the sum function $F$ of (1) is holomorphic in $A$ and the equality
\begin{align}
\frac{d^nF(z)}{dz^n} \;=\; F^{(n)}(z) \;=\; 
f_1^{(n)}(z)+f_2^{(n)}(z)+f_3^{(n)}(z)+\ldots
\end{align}
is true for every positive integer $n$ in all points of $A$.\, The series (2) converges uniformly in every compact subdomain of $A$.

\textbf{Theorem 3.}\, If $f(z)$ is holomorphic in a domain $A$ and $z_0$ is a point of $A$, then one can expand $f(z)$ to a power series (the so-called {\em Taylor series})
$$f(z) \;=\; \sum_{n=0}^\infty a_n(z\!-\!z_0)^n \quad \mathrm{where} 
\quad a_n \;=\; \frac{f^{(n)}(z_0)}{n!}\quad(n \,=\, 0,\,1,\,2,\,\ldots).$$
This \PMlinkescapetext{expansion} is valid at least in the greatest disk\, 
$|z-z_0| < r\; (\leqq \infty)$\, which contains points of $A$ only.

%%%%%
%%%%%
\end{document}
