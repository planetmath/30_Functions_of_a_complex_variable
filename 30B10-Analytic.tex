\documentclass[12pt]{article}
\usepackage{pmmeta}
\pmcanonicalname{Analytic}
\pmcreated{2013-03-22 12:04:36}
\pmmodified{2013-03-22 12:04:36}
\pmowner{djao}{24}
\pmmodifier{djao}{24}
\pmtitle{analytic}
\pmrecord{9}{31147}
\pmprivacy{1}
\pmauthor{djao}{24}
\pmtype{Definition}
\pmcomment{trigger rebuild}
\pmclassification{msc}{30B10}
\pmclassification{msc}{26A99}
\pmsynonym{real analytic}{Analytic}
\pmsynonym{real analytic function}{Analytic}
\pmsynonym{complex analytic}{Analytic}
\pmsynonym{complex analytic function}{Analytic}
\pmsynonym{analytic function}{Analytic}
\pmrelated{Holomorphic}
\pmrelated{TaylorSeries}
\pmrelated{CauchyKowalewskiTheorem}

\usepackage{amssymb}
\usepackage{amsmath}
\usepackage{amsfonts}
\usepackage{graphicx}
%%%\usepackage{xypic}
\begin{document}
Let $U$ be a domain in the complex numbers (resp., real numbers). A function $f: U \longrightarrow \mathbb{C}$ (resp., $f: U \longrightarrow \mathbb{R}$) is {\em analytic} (resp., {\em real analytic}) if $f$ has a Taylor series about each point $x \in U$ that converges to the function $f$ in an open neighborhood of $x$.

\section{On Analyticity and Holomorphicity}

A complex function is analytic if and only if it is holomorphic. Because of this equivalence, an analytic function in the complex case is often defined to be one that is holomorphic, instead of one having a Taylor series as above. Although the two definitions are equivalent, it is not an easy matter to prove their equivalence, and a reader who does not yet have this result available will have to pay attention as to which definition of analytic is being used.
%%%%%
%%%%%
%%%%%
\end{document}
