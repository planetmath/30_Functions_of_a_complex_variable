\documentclass[12pt]{article}
\usepackage{pmmeta}
\pmcanonicalname{MorerasTheorem}
\pmcreated{2013-03-22 12:58:09}
\pmmodified{2013-03-22 12:58:09}
\pmowner{matte}{1858}
\pmmodifier{matte}{1858}
\pmtitle{Morera's theorem}
\pmrecord{12}{33339}
\pmprivacy{1}
\pmauthor{matte}{1858}
\pmtype{Theorem}
\pmcomment{trigger rebuild}
\pmclassification{msc}{30D20}

\endmetadata

% this is the default PlanetMath preamble.  as your knowledge
% of TeX increases, you will probably want to edit this, but
% it should be fine as is for beginners.

% almost certainly you want these
\usepackage{amssymb}
\usepackage{amsmath}
\usepackage{amsfonts}

% used for TeXing text within eps files
%\usepackage{psfrag}
% need this for including graphics (\includegraphics)
%\usepackage{graphicx}
% for neatly defining theorems and propositions
%\usepackage{amsthm}
% making logically defined graphics
%%%\usepackage{xypic}

% there are many more packages, add them here as you need them

% define commands here

\newcommand{\sR}[0]{\mathbb{R}}
\newcommand{\sC}[0]{\mathbb{C}}
\newcommand{\sN}[0]{\mathbb{N}}
\newcommand{\sZ}[0]{\mathbb{Z}}
\begin{document}
Morera's theorem provides the converse of Cauchy's integral theorem.  

{\bf Theorem} \cite{rudin_real}
Suppose $G$ is a region in $\sC$, and $f:G\to \sC$ is a continuous 
function. If for every closed triangle $\Delta$ in $G$, we have 
$$\int_{\partial \Delta} f\, dz = 0,$$
then $f$ is analytic on $G$. (Here, $\partial \Delta$ is the piecewise linear
\PMlinkname{boundary}{BoundaryInTopology} of $\Delta$.)

In particular, if for every rectifiable closed curve $\Gamma$ in $G$, we have 
$\int_{\Gamma} f\, dz = 0,$
then $f$ is analytic on $G$. Proofs of this can be found most
undergraduate books on complex analysis \cite{kreyszig93, silverman}.

\begin{thebibliography}{9}
\bibitem{rudin_real}
 W. Rudin, \emph{Real and complex analysis}, 3rd ed., McGraw-Hill Inc., 1987.
 \bibitem {kreyszig93} E. Kreyszig,
 \emph{Advanced Engineering Mathematics},
 John Wiley \& Sons, 1993, 7th ed.
\bibitem{silverman}
 R.A. Silverman, \emph{Introductory Complex Analysis},
 Dover Publications, 1972.
 \end{thebibliography}
%%%%%
%%%%%
\end{document}
