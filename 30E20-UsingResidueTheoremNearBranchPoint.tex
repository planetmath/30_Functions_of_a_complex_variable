\documentclass[12pt]{article}
\usepackage{pmmeta}
\pmcanonicalname{UsingResidueTheoremNearBranchPoint}
\pmcreated{2013-03-22 18:42:50}
\pmmodified{2013-03-22 18:42:50}
\pmowner{pahio}{2872}
\pmmodifier{pahio}{2872}
\pmtitle{using residue theorem near branch point}
\pmrecord{15}{41480}
\pmprivacy{1}
\pmauthor{pahio}{2872}
\pmtype{Example}
\pmcomment{trigger rebuild}
\pmclassification{msc}{30E20}
\pmrelated{GeneralPower}
\pmrelated{ComplexLogarithm}
\pmrelated{ExampleOfUsingResidueTheorem}
\pmrelated{FresnelFormulas}
\pmrelated{ExampleOfChangingVariable}

\endmetadata

% this is the default PlanetMath preamble.  as your knowledge
% of TeX increases, you will probably want to edit this, but
% it should be fine as is for beginners.

% almost certainly you want these
\usepackage{amssymb}
\usepackage{amsmath}
\usepackage{amsfonts}

% used for TeXing text within eps files
%\usepackage{psfrag}
% need this for including graphics (\includegraphics)
%\usepackage{graphicx}
% for neatly defining theorems and propositions
 \usepackage{amsthm}
% making logically defined graphics
%%%\usepackage{xypic}
\usepackage{pstricks}
\usepackage{pst-plot}

% there are many more packages, add them here as you need them

% define commands here

\theoremstyle{definition}
\newtheorem*{thmplain}{Theorem}

\begin{document}
\PMlinkescapeword{cut}

Find the value of the improper integral
\begin{align}
\int_0^\infty\!\frac{x^{-k}}{x+1}\,dx \qquad (0 < k < 1)
\end{align}
by using the Cauchy residue theorem.\\

Since $k$ is not an integer,
a circuit around the origin changes the argument of 
$$z^{-k} \;=\; e^{-k\log z} \;=\; e^{-k(\ln|z|+i\arg{z})},$$
by the amount $-2k\pi$, thus giving to $z^{-k}$ a new value.\, Consequently the integrand 
$\displaystyle\frac{z^{-k}}{z+1}$ of (1) has the origin as a branch point.\, In the annulus between the origin-centered circles $\Gamma$ and $\gamma$, \PMlinkescapetext{cut open} along the positive real axis, any branch of the integrand is single-valued.\, Let's take the branch with\, $0 \leqq \arg{z} \leqq 2\pi$\, and suppose\, $0 < r < 1 < R$.\, Then the integrand function is analytic in the cut annulus except at the point \,$z = -1$\, which is a simple pole.

\begin{center}
\begin{pspicture}(-4.2,-4)(4.2,4)
\rput(-4.2,-4){.}
\rput(4.2,4){.}
\psaxes[Dx=10,Dy=10]{->}(0,0)(-4.2,-3.5)(4.2,3.5)
\pscircle[linewidth=0.05,linecolor=blue](0,0){0.75}
\pscircle[linewidth=0.05,linecolor=blue](0,0){3}
\psline[linewidth=0.08,linecolor=blue](0.75,0)(3,0)
\psline{->}(1.5,0.25)(2.5,0.25)
\psline{->}(2.5,-0.25)(1.5,-0.25)
\psline{->}(0.5,0.8)(0.8,0.5)
\psline{->}(2.5,2.1)(2.1,2.5)
\psdots[linecolor=blue](0.75,0)(3,0)
\psdot[linecolor=red](-1.5,0)
\psdot[linecolor=cyan](0,0)
\rput(0.85,-0.25){$r$}
\rput(3.15,-0.25){$R$}
\rput(-1.5,-0.25){$-1$}
\rput(-0.7,0.7){$\gamma$}
\rput(-2.5,2.5){$\Gamma$}
\end{pspicture}
\end{center}
By the residue theorem,
\begin{align}
\displaystyle\oint\!\frac{z^{-k}}{z\!+\!1}\,dz 
   \;=\; 2\pi i \operatorname{Res}\!\left(\frac{z^{-k}}{z\!+\!1};\,-1\right),
\end{align}
the integral taken around the border of the cut annulus.

One has first
$$\operatorname{Res}\!\left(\frac{z^{-k}}{z\!+\!1};\,-1\right) \;=\;
\lim_{z\to-1}(z+1)\frac{z^{-k}}{z\!+\!1} \;=\; (-1)^{-k} \;=\; e^{-i\pi k}.$$

The integral (2) is split to four parts, two of them are the integrals around $\Gamma$ and $\gamma$, and the other two the integrals along the cut on the real axis, in two directions.\, Using the estimating theorem of contour integral, one obtains
$$\left|\int_\Gamma\!\frac{z^{-k}}{z\!+\!1}\,dz\right| \;\leqq\; \frac{R^{-k}}{R\!-\!1}\cdot2\pi R 
\;=\; \frac{2\pi R^{-k}}{1\!-\!\frac{1}{R}} \;\to\; 0 \quad \mbox{as} \quad R \to \infty,$$
$$\left|\int_\gamma\!\frac{z^{-k}}{z\!+\!1}\,dz\right| \;\leqq\; \frac{r^{-k}}{1\!-\!r}\cdot2\pi r 
\;=\; \frac{2\pi r^{1-k}}{1\!-\!r} \;\to\; 0 \,\quad \mbox{as} \,\quad r \to 0\!+\!.$$
Thus the two first parts of the integral (2) have the limits 0 as\, $R \to \infty$\, and\, $r \to 0\!+$.\, When one integrates on the real axis from $R$ to $r$ along the ``lower edge'' of the cut, denoting\, $z = x\!+\!iy$ ($x,\,y \in \mathbb{R}$), one has\, 
$\arg{z} = 2\pi$\, and
$$z^{-k} \;=\; e^{-k(\ln{x}+i\cdot2\pi)} \;=\; x^{-k}e^{-2i\pi k},$$
whence
$$\int_R^r\!\frac{z^{-k}}{z\!+\!1}\,dz \;=\; e^{-2i\pi k}\int_R^r\!\frac{x^{-k}}{x\!+\!1}\,dx 
\;=\; -e^{-2i\pi k}\int_r^R\!\frac{x^{-k}}{x\!+\!1}\,dx.$$
But the last written integral equals to the integral gotten as one integrates from $r$ to $R$ along the ``upper edge'' of the cut, because on this path one has\, $\arg{z} = 0$\, and thus\, $z^{-k} = x^{-k}$.\, Accordingly, the sum of the integrals taken along the cut has the value
$$\left(1-e^{-2i\pi k}\right)\!\int_r^R \frac{x^{-k}}{x\!+\!1}\,dx.$$
Accordingly, (2) gives the limit equation
$$\left(1-e^{-2i\pi k}\right)\!\int_0^\infty\!\frac{x^{-k}}{x\!+\!1}\,dx \;=\; 2i\pi e^{-i\pi k},$$
and this implies moreover
$$\int_0^\infty\!\frac{x^{-k}}{x\!+\!1}\,dx \;=\; 2i\pi\frac{e^{-i\pi k}}{1-e^{-2i\pi k}} 
\;=\; \frac{2i\pi}{e^{i\pi k}\!-\!e^{-i\pi k}}.$$
Thus the \PMlinkname{Euler formula}{ComplexSineAndCosine} lastly yields the result
\begin{align}
\int_0^\infty\!\frac{x^{-k}}{x\!+\!1}\,dx \;=\; \frac{\pi}{\sin{\pi k}} \qquad (0 \,<\, k \,<\, 1).
\end{align}
%%%%%
%%%%%
\end{document}
