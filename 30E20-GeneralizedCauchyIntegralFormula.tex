\documentclass[12pt]{article}
\usepackage{pmmeta}
\pmcanonicalname{GeneralizedCauchyIntegralFormula}
\pmcreated{2013-03-22 17:46:41}
\pmmodified{2013-03-22 17:46:41}
\pmowner{jirka}{4157}
\pmmodifier{jirka}{4157}
\pmtitle{generalized Cauchy integral formula}
\pmrecord{6}{40236}
\pmprivacy{1}
\pmauthor{jirka}{4157}
\pmtype{Theorem}
\pmcomment{trigger rebuild}
\pmclassification{msc}{30E20}
\pmsynonym{generalized Cauchy formula}{GeneralizedCauchyIntegralFormula}

\endmetadata

% this is the default PlanetMath preamble.  as your knowledge
% of TeX increases, you will probably want to edit this, but
% it should be fine as is for beginners.

% almost certainly you want these
\usepackage{amssymb}
\usepackage{amsmath}
\usepackage{amsfonts}

% used for TeXing text within eps files
%\usepackage{psfrag}
% need this for including graphics (\includegraphics)
%\usepackage{graphicx}
% for neatly defining theorems and propositions
\usepackage{amsthm}
% making logically defined graphics
%%%\usepackage{xypic}

% there are many more packages, add them here as you need them

% define commands here
\theoremstyle{theorem}
\newtheorem*{thm}{Theorem}
\newtheorem*{lemma}{Lemma}
\newtheorem*{conj}{Conjecture}
\newtheorem*{cor}{Corollary}
\newtheorem*{example}{Example}
\newtheorem*{prop}{Proposition}
\theoremstyle{definition}
\newtheorem*{defn}{Definition}
\theoremstyle{remark}
\newtheorem*{rmk}{Remark}

\begin{document}
\begin{thm}
Let $U \subset \mathbb{C}$ be a domain with $C^1$ boundary.  Let $f \colon U \to \mathbb{C}$
be a $C^1$ function that is $C^1$ up to the boundary.  Then for $z \in U,$
\begin{equation*}
f(z) =
\frac{1}{2\pi i}
\int_{\partial U} \frac{f(w)}{w-z} dw
-
\frac{1}{2\pi i}
\int_{U} \frac{\frac{\partial f}{\partial \bar{z}}(w)}{w-z} d\bar{w} \wedge dw .
\end{equation*}
\end{thm}

Note that $C^1$ up to the boundary means that the function and the derivative extend to be continuous
functions on the closure of $U.$  The theorem follows from Stokes' theorem.  When $f$ is holomorphic,
then the second term is zero and this is the classical Cauchy integral formula.

\begin{thebibliography}{9}
\bibitem{Hormander:several}
Lars H\"ormander.
{\em \PMlinkescapetext{An Introduction to Complex Analysis in Several
Variables}},
North-Holland Publishing Company, New York, New York, 1973.
\bibitem{Krantz:several}
Steven~G.\@ Krantz.
{\em \PMlinkescapetext{Function Theory of Several Complex Variables}},
AMS Chelsea Publishing, Providence, Rhode Island, 1992.
\end{thebibliography}
%%%%%
%%%%%
\end{document}
