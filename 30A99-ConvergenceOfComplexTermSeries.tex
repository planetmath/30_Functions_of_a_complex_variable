\documentclass[12pt]{article}
\usepackage{pmmeta}
\pmcanonicalname{ConvergenceOfComplexTermSeries}
\pmcreated{2014-10-31 19:04:59}
\pmmodified{2014-10-31 19:04:59}
\pmowner{pahio}{2872}
\pmmodifier{pahio}{2872}
\pmtitle{convergence of complex term series}
\pmrecord{8}{41451}
\pmprivacy{1}
\pmauthor{pahio}{2872}
\pmtype{Theorem}
\pmcomment{trigger rebuild}
\pmclassification{msc}{30A99}
\pmclassification{msc}{40A05}
\pmrelated{OrderOfFactorsInInfiniteProduct}

\endmetadata

% this is the default PlanetMath preamble.  as your knowledge
% of TeX increases, you will probably want to edit this, but
% it should be fine as is for beginners.

% almost certainly you want these
\usepackage{amssymb}
\usepackage{amsmath}
\usepackage{amsfonts}

% used for TeXing text within eps files
%\usepackage{psfrag}
% need this for including graphics (\includegraphics)
%\usepackage{graphicx}
% for neatly defining theorems and propositions
 \usepackage{amsthm}
% making logically defined graphics
%%%\usepackage{xypic}

% there are many more packages, add them here as you need them

% define commands here

\theoremstyle{definition}
\newtheorem*{thmplain}{Theorem}

\begin{document}
\PMlinkescapeword{terms}
A series
\begin{align}
\sum_{\nu=1}^\infty c_\nu \;=\; c_1\!+\!c_2\!+\!c_3\!+\ldots
\end{align}
with complex terms
$$c_\nu \;=\; a_\nu\!+\!ib_\nu \qquad (a_\nu,\,b_\nu \in \mathbb{R}\;\; \forall\, \nu)$$
is convergent iff the sequence of its partial sums converges to a complex number.\\

\textbf{Theorem 1.}\, The series (1) converges iff the series
\begin{align}
\sum_{\nu=1}^\infty a_\nu \quad \mbox{and} \quad \sum_{\nu=1}^\infty b_\nu
\end{align}
formed by real parts and the imaginary parts of its terms both are convergent.

{\em Proof.}\, Let\, $\varepsilon > 0$.\, Denote\, 
$$\sum_{\nu=1}^n a_\nu \,:=\, s_n, \quad \sum_{\nu=1}^n b_\nu \,:=\, t_n, \quad \sum_{\nu=1}^n c_\nu \,:=\, u_n.$$
If the series (2) are convergent with sums $S$ and $T$, then there is a number $N$ such that
$$|s_n-S| < \frac{\varepsilon}{2}, \quad |t_n-T| < \frac{\varepsilon}{2} \quad \mbox{when} \quad n \geqq N.$$
Accordingly,
$$|u_n-(S\!+\!iT)| = \sqrt{(s_n-S)^2+(t_n-T)^2} \leqq |s_n-S|+|t_n-T| < \varepsilon \quad \mbox{when} \quad n \geqq N,$$
i.e. the series (1) converges to $S\!+\!iT$.\, If, conversely, (1) converges to a complex number\, 
$$u \,=\, s\!+\!it \quad (s,\,t \in\mathbb{R}),$$
then\, 
$$|s_n-s| \,\leqq\, |(s_n-s)+i(t_n-t)| \,=\, |u_n-u|, \quad |t_n-t| \,\leqq\, |(s_n-s)+i(t_n-t)| \,=\, |u_n-u|,$$
and consequently,\, $\displaystyle\lim_{n\to\infty}s_n \,=\, s$\, and\, $\displaystyle\lim_{n\to\infty}t_n \,=\, t$, i.e. the series (2) are convergent with sums the real numbers $s$ and $t$.\\

\textbf{Theorem 2.}\, The series (1) converges absolutely iff the series (2) both converge absolutely.

{\em Proof.}\, The absolute convergence of (1) means that the series
$$\sum_{\nu=1}^\infty |c_\nu|$$
converges.\, But since\, $|c_\nu|^2 \,=\, |a_\nu|^2+|b_\nu|^2$,\, we have
$$|a_\nu| \,\leqq\, |c_\nu|; \quad |b_\nu| \,\leqq\, |c_\nu| \,\leqq\, |a_\nu|+|c_\nu|.$$
From these inequalities we can infer the assertion of the theorem 2.

\begin{thebibliography}{9}
\bibitem{NP}{\sc R. Nevanlinna \& V. Paatero}: {\em Funktioteoria}.\, Kustannusosakeyhti\"o Otava. Helsinki (1963).
\end{thebibliography}



%%%%%
%%%%%
\end{document}
