\documentclass[12pt]{article}
\usepackage{pmmeta}
\pmcanonicalname{UnivalentAnalyticFunction}
\pmcreated{2013-03-22 14:12:06}
\pmmodified{2013-03-22 14:12:06}
\pmowner{jirka}{4157}
\pmmodifier{jirka}{4157}
\pmtitle{univalent analytic function}
\pmrecord{6}{35633}
\pmprivacy{1}
\pmauthor{jirka}{4157}
\pmtype{Definition}
\pmcomment{trigger rebuild}
\pmclassification{msc}{30C55}
\pmsynonym{univalent function}{UnivalentAnalyticFunction}
\pmsynonym{univalent}{UnivalentAnalyticFunction}

\endmetadata

% this is the default PlanetMath preamble.  as your knowledge
% of TeX increases, you will probably want to edit this, but
% it should be fine as is for beginners.

% almost certainly you want these
\usepackage{amssymb}
\usepackage{amsmath}
\usepackage{amsfonts}

% used for TeXing text within eps files
%\usepackage{psfrag}
% need this for including graphics (\includegraphics)
%\usepackage{graphicx}
% for neatly defining theorems and propositions
\usepackage{amsthm}
% making logically defined graphics
%%%\usepackage{xypic}

% there are many more packages, add them here as you need them

% define commands here
\theoremstyle{theorem}
\newtheorem*{thm}{Theorem}
\newtheorem*{lemma}{Lemma}
\newtheorem*{conj}{Conjecture}
\newtheorem*{cor}{Corollary}
\newtheorem*{example}{Example}
\newtheorem*{prop}{Proposition}
\theoremstyle{definition}
\newtheorem*{defn}{Definition}
\theoremstyle{remark}
\newtheorem*{rmk}{Remark}
\begin{document}
\begin{defn}
An analytic function on an open set is called {\em univalent} if it is one-to-one.
\end{defn}

For example mappings of the unit disc to itself $\phi_a : {\mathbb{D}} \rightarrow {\mathbb{D}}$, where
$\phi_a(z) = \frac{z-a}{1 - \bar{a}z}$, for any $a \in {\mathbb{D}}$ are univalent.
The following \PMlinkescapetext{proposition} summarizes some
basic \PMlinkescapetext{properties} of univalent functions.

\begin{prop}
Suppose that $G,\Omega \subset {\mathbb{C}}$ are regions and
$f \colon G \to \Omega$ is a univalent mapping such that $f(G) = \Omega$ (it
is onto), then
\begin{itemize}
\item $f^{-1} \colon \Omega \to G$ (where $f^{-1}(f(z)) = z$) is an analytic
function and $(f^{-1})'(f(z)) = \frac{1}{f'(z)}$,
\item $f'(z) \not= 0$ for all $z \in G$
\end{itemize}
\end{prop}

\begin{thebibliography}{9}
\bibitem{Conway:complexI}
John~B. Conway.
{\em \PMlinkescapetext{Functions of One Complex Variable I}}.
Springer-Verlag, New York, New York, 1978.
\bibitem{Conway:complexII}
John~B. Conway.
{\em \PMlinkescapetext{Functions of One Complex Variable II}}.
Springer-Verlag, New York, New York, 1995.
\end{thebibliography}
%%%%%
%%%%%
\end{document}
