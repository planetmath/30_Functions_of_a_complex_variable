\documentclass[12pt]{article}
\usepackage{pmmeta}
\pmcanonicalname{TwoSeriesArisingFromTheAlternatingZetaFunction}
\pmcreated{2013-06-06 19:13:30}
\pmmodified{2013-06-06 19:13:30}
\pmowner{pahio}{2872}
\pmmodifier{pahio}{2872}
\pmtitle{two series arising from the alternating zeta function}
\pmrecord{10}{42581}
\pmprivacy{1}
\pmauthor{pahio}{2872}
\pmtype{Conjecture}
\pmcomment{trigger rebuild}
\pmclassification{msc}{30D99}
\pmclassification{msc}{30B50}
\pmclassification{msc}{11M41}
\pmsynonym{trigonometric series conjecture equivalent to the Riemann hypothesis}{TwoSeriesArisingFromTheAlternatingZetaFunction}
%\pmkeywords{Riemann hypothesis}
\pmrelated{EulerRelation}

\endmetadata

% this is the default PlanetMath preamble.  as your knowledge
% of TeX increases, you will probably want to edit this, but
% it should be fine as is for beginners.

% almost certainly you want these
\usepackage{amssymb}
\usepackage{amsmath}
\usepackage{amsfonts}

% used for TeXing text within eps files
%\usepackage{psfrag}
% need this for including graphics (\includegraphics)
%\usepackage{graphicx}
% for neatly defining theorems and propositions
 \usepackage{amsthm}
% making logically defined graphics
%%%\usepackage{xypic}

% there are many more packages, add them here as you need them

% define commands here

\theoremstyle{definition}
\newtheorem*{thmplain}{Theorem}

\begin{document}
The terms of the series defining the alternating zeta function
$$\eta(s) \;:=\; \sum_{n=1}^{\infty}\frac{(-1)^{n-1}}{n^s} \qquad (\mbox{Re}\,s > 0),$$
a.k.a. the Dirichlet eta function, may be split into their real and imaginary parts:
$$\frac{1}{n^s} \;=\; \frac{e^{-ib\ln{n}}}{n^a} \;=\; 
\frac{\cos(b\ln{n})}{n^a}-\frac{i\sin(b\ln{n})}{n^a}$$
Here,\, $s = a\!+\!ib$\, with real $a$ and $b$.\, It follows the equation
\begin{align}
\eta(s) \;=\; 
-\sum_{n=1}^\infty\frac{(-1)^n}{n^a}\cos(b\ln{n})+i\sum_{n=1}^\infty\frac{(-1)^n}{n^a}\sin(b\ln{n})
\end{align}
containing two Dirichlet series.

The alternating zeta function and the Riemann zeta function are connected by the relation
$$\zeta(s) \;=\; \frac{\eta(s)}{1-2^{1-s}}$$
(see the \PMlinkname{parent entry}{AnalyticContinuationOfRiemannZetaToCriticalStrip}).\, The following conjecture concerning the above real part series and imaginary part series of (1) has been proved by Sondow [1] to be equivalent with the Riemann hypothesis.\\

\textbf{Conjecture.}\, If the equations
$$\sum_{n=1}^\infty\frac{(-1)^n}{n^a}\cos(b\ln{n}) \;=\; 0 \quad \mbox{and} \quad 
\sum_{n=1}^\infty\frac{(-1)^n}{n^a}\sin(b\ln{n}) \;=\; 0$$
are true for some pair of real numbers $a$ and $b$, then
$$a \;=\; 1/2 \qquad \mbox{or} \qquad a \;=\; 1.$$

\begin{thebibliography}{8}
\bibitem{S}{\sc Jonathan Sondow}: A simple counterexample to Havil's ``reformulation'' of the Riemann hypothesis.\, -- {\it Elemente der Mathematik} \textbf{67} (2012) 61--67.\, Also available \PMlinkexternal{here}{http://arxiv.org/pdf/0706.2840v3.pdf}.
\end{thebibliography}

%%%%%
%%%%%
\end{document}
