\documentclass[12pt]{article}
\usepackage{pmmeta}
\pmcanonicalname{DigammaAndPolygammaFunction}
\pmcreated{2013-03-22 15:53:21}
\pmmodified{2013-03-22 15:53:21}
\pmowner{rspuzio}{6075}
\pmmodifier{rspuzio}{6075}
\pmtitle{digamma and polygamma function}
\pmrecord{13}{37890}
\pmprivacy{1}
\pmauthor{rspuzio}{6075}
\pmtype{Definition}
\pmcomment{trigger rebuild}
\pmclassification{msc}{30D30}
\pmclassification{msc}{33B15}
\pmdefines{digamma function}
\pmdefines{polygamma function}

\endmetadata

% this is the default PlanetMath preamble.  as your knowledge
% of TeX increases, you will probably want to edit this, but
% it should be fine as is for beginners.

% almost certainly you want these
\usepackage{amssymb}
\usepackage{amsmath}
\usepackage{amsfonts}

% used for TeXing text within eps files
%\usepackage{psfrag}
% need this for including graphics (\includegraphics)
%\usepackage{graphicx}
% for neatly defining theorems and propositions
%\usepackage{amsthm}
% making logically defined graphics
%%%\usepackage{xypic}

% there are many more packages, add them here as you need them

% define commands here
\begin{document}
The \emph{digamma function} is defined as the logarithmic derivative of the gamma function:
 \[ \psi (z) = {d \over dz} \log \Gamma (z) = {\Gamma' (z) \over \Gamma (z)}. \]
Likewise the \emph{polygamma functions} are defined as higher order logarithmic derivatives of the gamma function:
 \[ \psi^{(n)} (z) = {d^n \over dz^n} \log \Gamma (z). \]

These equations enjoy functional equations which are closely related to those of the gamma function:
\begin{align*}
\psi (z+1) &= \psi (z) + {1 \over z} \\
\psi (1-z) &= \psi (z) + \pi \cot \pi z \\
\psi (2z)  &= {1 \over 2} \psi (z) + {1 \over 2} \psi \left( z + {1 \over 2} 
\right) + \log 2 \\
\psi^{(n)} (z+1) &= \psi^{(n)} (z) + (-1)^n {n! \over z^{n+1}} \\
\end{align*}

These functions have poles at the negative integers and can be expressed as partial fraction series:
\begin{align}
\psi(z) = -\gamma-{1\over z}+\sum_{k=1}^\infty \left({1\over k}-{1\over z\!+\!k}\right), 
\end{align}
\begin{align}
\psi^{(n)}(z) = (-1)^nn!\!\sum_{k=0}^\infty{1\over(z\!+\!k)^n}
\end{align}
Here, $\gamma$ is \PMlinkname{Euler--Mascheroni constant}{EulerMascheroniConstant}.\, Substituting\, $z = 1$\, to (1), one gets the value
$$\Gamma\,'(1) \;=\; -\gamma.$$
%%%%%
%%%%%
\end{document}
