\documentclass[12pt]{article}
\usepackage{pmmeta}
\pmcanonicalname{MobiusTransformationCrossratioPreservationTheorem}
\pmcreated{2013-03-22 13:35:50}
\pmmodified{2013-03-22 13:35:50}
\pmowner{rspuzio}{6075}
\pmmodifier{rspuzio}{6075}
\pmtitle{M\"obius transformation cross-ratio preservation theorem}
\pmrecord{9}{34222}
\pmprivacy{1}
\pmauthor{rspuzio}{6075}
\pmtype{Theorem}
\pmcomment{trigger rebuild}
\pmclassification{msc}{30E20}
\pmrelated{CrossRatio}

% this is the default PlanetMath preamble.  as your knowledge
% of TeX increases, you will probably want to edit this, but
% it should be fine as is for beginners.

% almost certainly you want these
\usepackage{amssymb}
\usepackage{amsmath}
\usepackage{amsfonts}

% used for TeXing text within eps files
%\usepackage{psfrag}
% need this for including graphics (\includegraphics)
%\usepackage{graphicx}
% for neatly defining theorems and propositions
%\usepackage{amsthm}
% making logically defined graphics
%%%\usepackage{xypic}

% there are many more packages, add them here as you need them

% define commands here
\begin{document}
A M\"obius transformation $f: z \mapsto w$ preserves the cross-ratios, i.e. \\
\begin{displaymath}
\frac{(w_1-w_2)(w_3-w_4)}{(w_1-w_4)(w_3-w_2)} = \frac{(z_1-z_2)(z_3-z_4)}{(z_1-z_4)(z_3-z_2)}
\end{displaymath}

Conversely, given two quadruplets which have the same cross-ratio, there
exists a M\"obius transformation which maps one quadruplet to the other.

A consequence of this result is that the cross-ratio of $(a,b,c,d)$ is the
value at $a$ of the M\"obius transformation that takes $b$, $c$, $d$, to
$1$, $0$, $\infty$ respectively.
%%%%%
%%%%%
\end{document}
