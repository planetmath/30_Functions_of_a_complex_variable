\documentclass[12pt]{article}
\usepackage{pmmeta}
\pmcanonicalname{HurwitzsTheorem}
\pmcreated{2013-03-22 14:17:55}
\pmmodified{2013-03-22 14:17:55}
\pmowner{jirka}{4157}
\pmmodifier{jirka}{4157}
\pmtitle{Hurwitz's theorem}
\pmrecord{5}{35756}
\pmprivacy{1}
\pmauthor{jirka}{4157}
\pmtype{Theorem}
\pmcomment{trigger rebuild}
\pmclassification{msc}{30C15}
\pmrelated{CompositionAlgebraOverAlgebaicallyClosedFields}
\pmrelated{CompositionAlgebrasOverMathbbR}
\pmrelated{CompositionAlgebrasOverFiniteFields}
\pmrelated{CompositionAlgebrasOverMathbbQ}

% this is the default PlanetMath preamble.  as your knowledge
% of TeX increases, you will probably want to edit this, but
% it should be fine as is for beginners.

% almost certainly you want these
\usepackage{amssymb}
\usepackage{amsmath}
\usepackage{amsfonts}

% used for TeXing text within eps files
%\usepackage{psfrag}
% need this for including graphics (\includegraphics)
%\usepackage{graphicx}
% for neatly defining theorems and propositions
\usepackage{amsthm}
% making logically defined graphics
%%%\usepackage{xypic}

% there are many more packages, add them here as you need them

% define commands here
\theoremstyle{theorem}
\newtheorem*{thm}{Theorem}
\newtheorem*{lemma}{Lemma}
\newtheorem*{conj}{Conjecture}
\newtheorem*{cor}{Corollary}
\newtheorem*{example}{Example}
\theoremstyle{definition}
\newtheorem*{defn}{Definition}
\begin{document}
Define the ball at $z_0$ of radius $r$ as $B(z_0,r) = \{ z \in G : \lvert z-z_0 \rvert < r \}$ and $D(z_0,r) = \{ z \in G : \lvert z-z_0 \rvert \leq r \}$ is the closed ball at $z_0$ of radius $r$.

\begin{thm}[Hurwitz]
Let $G \subset {\mathbb{C}}$ be a region and suppose the sequence
of holomorphic functions $\{ f_n \}$ converges uniformly on compact subsets
of $G$ to a holomorphic function $f$.  If $f$ is not identically zero,
$D(z_0,r) \subset G$ and $f(z) \not= 0$ for $z$ such that
$\lvert z-z_0 \rvert = r$, then there exists an $N$ such that for all $n \geq N$
$f$ and $f_n$ have the same number of zeros in $B(z_0,r)$.
\end{thm}

What this theorem says is that if you have a sequence of holomorphic functions which converge uniformly on compact subsets (such a sequence always converges to a holomorphic function but that's another theorem altogether), the \PMlinkescapetext{limit}
function is not identically zero and furthermore the \PMlinkescapetext{limit} function is not
zero on the boundary of some ball,
then eventually 
the functions of the sequence have the same number of zeros inside this ball as does the \PMlinkescapetext{limit} function.

Do note the requirement for $f$ not being identically zero.  For example the sequence $f_n(z) := \frac{1}{n}$ converges uniformly on compact subsets to
$f(z) := 0$, but $f_n$ have no zeros anywhere, while $f$ is identically zero.

Also in general this result holds for bounded \PMlinkname{convex subsets}{ConvexSet} but it is most
useful to \PMlinkescapetext{state} for balls.

An immediate consequence of this theorem is this useful corollary.

\begin{cor}
If $G$ is a region and a sequence of holomorphic functions $\{ f_n \}$ converges
uniformly on compact subsets of $G$ to a holomorphic function $f$, and furthermore if $f_n$ never vanishes (is not zero for any point in $G$), then
$f$ is either identically zero or also never vanishes.
\end{cor}

\begin{thebibliography}{9}
\bibitem{Conway:complexI}
John~B. Conway.
{\em \PMlinkescapetext{Functions of One Complex Variable I}}.
Springer-Verlag, New York, New York, 1978.
\end{thebibliography}
%%%%%
%%%%%
\end{document}
