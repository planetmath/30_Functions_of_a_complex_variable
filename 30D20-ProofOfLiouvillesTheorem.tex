\documentclass[12pt]{article}
\usepackage{pmmeta}
\pmcanonicalname{ProofOfLiouvillesTheorem}
\pmcreated{2013-03-22 12:54:15}
\pmmodified{2013-03-22 12:54:15}
\pmowner{Evandar}{27}
\pmmodifier{Evandar}{27}
\pmtitle{proof of Liouville's theorem}
\pmrecord{5}{33254}
\pmprivacy{1}
\pmauthor{Evandar}{27}
\pmtype{Proof}
\pmcomment{trigger rebuild}
\pmclassification{msc}{30D20}

\endmetadata

% this is the default PlanetMath preamble.  as your knowledge
% of TeX increases, you will probably want to edit this, but
% it should be fine as is for beginners.

% almost certainly you want these
\usepackage{amssymb}
\usepackage{amsmath}
\usepackage{amsfonts}

% used for TeXing text within eps files
%\usepackage{psfrag}
% need this for including graphics (\includegraphics)
%\usepackage{graphicx}
% for neatly defining theorems and propositions
%\usepackage{amsthm}
% making logically defined graphics
%%%\usepackage{xypic} 

% there are many more packages, add them here as you need them

% define commands here
\begin{document}
Let $f:\mathbb{C}\to\mathbb{C}$ be a bounded, entire function.  Then by Taylor's theorem, $$f(z)=\sum_{n=0}^\infty c_n x^n \mbox{ where } c_n = \frac{1}{2\pi i}\int_{\Gamma_r} \frac{f(w)}{w^{n+1}}\,dw$$

where $\Gamma_r$ is the circle of radius $r$ about $0$, for $r>0$.  Then $c_n$ can be estimated as $$|c_n|\leq\frac{1}{2\pi}\operatorname{length}(\Gamma_r)\operatorname{sup}\left\{\left|\frac{f(w)}{w^{n+1}}\right|\,\colon\, w\in \Gamma_r\right\} = \frac{1}{2\pi}\, 2\pi r \frac{M_r}{r^{n+1}} = \frac{M_r}{r^{n}}$$

where $M_r = \operatorname{sup}\{|f(w)|\colon w\in\Gamma_r\}$.

But $f$ is bounded, so there is $M$ such that $M_r\leq M$ for all $r$.  Then $|c_n|\leq\frac{M}{r^n}$ for all $n$ and all $r>0$.  But since $r$ is arbitrary, this gives $c_n = 0$ whenever $n>0$.  So $f(z) = c_0$ for all $z$, so $f$ is constant.$\square$
%%%%%
%%%%%
\end{document}
