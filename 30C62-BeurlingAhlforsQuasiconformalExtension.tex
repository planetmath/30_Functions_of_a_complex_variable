\documentclass[12pt]{article}
\usepackage{pmmeta}
\pmcanonicalname{BeurlingAhlforsQuasiconformalExtension}
\pmcreated{2013-03-22 14:06:49}
\pmmodified{2013-03-22 14:06:49}
\pmowner{jirka}{4157}
\pmmodifier{jirka}{4157}
\pmtitle{Beurling-Ahlfors quasiconformal extension}
\pmrecord{12}{35517}
\pmprivacy{1}
\pmauthor{jirka}{4157}
\pmtype{Theorem}
\pmcomment{trigger rebuild}
\pmclassification{msc}{30C62}
\pmsynonym{Beurling-Ahlfors theorem}{BeurlingAhlforsQuasiconformalExtension}
\pmrelated{QuasiconformalMapping}
\pmrelated{QuasisymmetricMapping}

\endmetadata

% this is the default PlanetMath preamble.  as your knowledge
% of TeX increases, you will probably want to edit this, but
% it should be fine as is for beginners.

% almost certainly you want these
\usepackage{amssymb}
\usepackage{amsmath}
\usepackage{amsfonts}

% used for TeXing text within eps files
%\usepackage{psfrag}
% need this for including graphics (\includegraphics)
%\usepackage{graphicx}
% for neatly defining theorems and propositions
\usepackage{amsthm}
% making logically defined graphics
%%%\usepackage{xypic}

% there are many more packages, add them here as you need them

% define commands here
\theoremstyle{theorem}
\newtheorem*{thm}{Theorem}
\newtheorem*{lemma}{Lemma}
\newtheorem*{conj}{Conjecture}
\newtheorem*{cor}{Corollary}
\theoremstyle{definition}
\newtheorem*{defn}{Definition}
\begin{document}
\begin{thm}[Beurling-Ahlfors]
There exists a quasiconformal mapping of the upper half plane to itself if and only if the boundary correspondence mapping $\mu$ is \PMlinkname{$M$-quasisymmetric}{QuasisymmetricMapping}.  Furthermore there exists an extension of $\mu$ to a quasiconformal mapping of the upper half planes such that the maximal dilatation of the extension depends only on $M$ and not on $\mu$.
\end{thm}

That is, the extension is \PMlinkname{$K$-quasiconformal}{QuasiconformalMapping} if and only if the boundary correspondence is \PMlinkname{$M$-quasisymmetric}{QuasisymmetricMapping} where $K$ depends purely on $M$.

Supposing that we have the mapping $\phi: H \rightarrow H$ (where $H$ is the upper half plane), then the mapping $\mu : {\mathbb{R}} \rightarrow {\mathbb{R}}$,
such that $\mu(x) = \phi(x)$ where $x \in {\mathbb{R}}$, is the boundary correspondence of $\phi$.

To prove the sufficiency of the above theorem Beurling and Ahlfors \cite{ba1956} define $\phi$ as follows.  Given a $\mu$ that is a quasisymmetric mapping of the real line onto itself and fixes $\infty$, we define a map $\phi(x,y) = u(x,y) + iu(x,y)$ where
\begin{align*}
u(x,y) & = \frac{1}{2y} \int_{-y}^y \mu(x+t) dt ,
\\
v(x,y) & = \frac{1}{2y} \int_0^y (\mu(x+t) - \mu(x-t)) dt .
\end{align*}

Intuitively $\phi$ is a function which ``smoothes'' out any kinks in the function $\mu$ as we get further and further away from the real line.  It therefore intuitively follows that $\phi$ has the worst (highest) dilatation near the $x$ axis, which actually turns out to be true.

\begin{thebibliography}{9}
\bibitem{ahlfors}
L.\@ V.\@ Ahlfors.  \emph{\PMlinkescapetext{Lectures on Quasiconformal
Mappings}}.  Van Nostrand-Reinhold, Princeton, New Jersey, 1966
\bibitem{ba1956}
A.\@ Beurling, L\@. V.\@ Ahlfors.  \PMlinkescapetext{The boundary
correspondence under quasiconformal mappings}.  \emph{Acta Math.}, 96:125-142,
1956.
\bibitem{lebl2003}
J.\@ Lebl.  \emph{\PMlinkescapetext{Quasiconformal Extensions of Quasisymmetric
Mappings}}.  \PMlinkescapetext{Masters thesis, San Diego State University, San
Diego, CA, May 2003}.  Also available at
\PMlinkexternal{http://www.jirka.org/thesis.pdf}{http://www.jirka.org/thesis.pdf}
\end{thebibliography}
%%%%%
%%%%%
\end{document}
