\documentclass[12pt]{article}
\usepackage{pmmeta}
\pmcanonicalname{HardysTheorem}
\pmcreated{2013-03-22 14:19:44}
\pmmodified{2013-03-22 14:19:44}
\pmowner{jirka}{4157}
\pmmodifier{jirka}{4157}
\pmtitle{Hardy's theorem}
\pmrecord{6}{35798}
\pmprivacy{1}
\pmauthor{jirka}{4157}
\pmtype{Theorem}
\pmcomment{trigger rebuild}
\pmclassification{msc}{30C80}
\pmclassification{msc}{30E20}
\pmrelated{HadamardThreeCircleTheorem}

\endmetadata

% this is the default PlanetMath preamble.  as your knowledge
% of TeX increases, you will probably want to edit this, but
% it should be fine as is for beginners.

% almost certainly you want these
\usepackage{amssymb}
\usepackage{amsmath}
\usepackage{amsfonts}

% used for TeXing text within eps files
%\usepackage{psfrag}
% need this for including graphics (\includegraphics)
%\usepackage{graphicx}
% for neatly defining theorems and propositions
\usepackage{amsthm}
% making logically defined graphics
%%%\usepackage{xypic}

% there are many more packages, add them here as you need them

% define commands here
\theoremstyle{theorem}
\newtheorem*{thm}{Theorem}
\newtheorem*{lemma}{Lemma}
\newtheorem*{conj}{Conjecture}
\newtheorem*{cor}{Corollary}
\newtheorem*{example}{Example}
\theoremstyle{definition}
\newtheorem*{defn}{Definition}
\begin{document}
\begin{thm}
Let $f$ be a holomorphic function on $B(0,R)$ (the open ball of radius $R$)
and $f$ is not a constant function, then
\begin{equation*}
I(r) := \frac{1}{2\pi} \int_0^{2\pi} \lvert f(r e^{i\theta}) \rvert d\theta
\end{equation*}
is strictly increasing and $\log I(r)$
is a convex function of $\log r$.
\end{thm}

\begin{thebibliography}{9}
\bibitem{Conway:complexI}
John~B. Conway.
{\em \PMlinkescapetext{Functions of One Complex Variable I}}.
Springer-Verlag, New York, New York, 1978.
\end{thebibliography}
%%%%%
%%%%%
\end{document}
