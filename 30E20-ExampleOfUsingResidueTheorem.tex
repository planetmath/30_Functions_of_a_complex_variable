\documentclass[12pt]{article}
\usepackage{pmmeta}
\pmcanonicalname{ExampleOfUsingResidueTheorem}
\pmcreated{2013-03-22 15:19:30}
\pmmodified{2013-03-22 15:19:30}
\pmowner{pahio}{2872}
\pmmodifier{pahio}{2872}
\pmtitle{example of using residue theorem}
\pmrecord{19}{37136}
\pmprivacy{1}
\pmauthor{pahio}{2872}
\pmtype{Example}
\pmcomment{trigger rebuild}
\pmclassification{msc}{30E20}
\pmrelated{ImproperIntegral}
\pmrelated{Residue}
\pmrelated{IntegralsOfEvenAndOddFunctions}
\pmrelated{UsingResidueTheoremNearBranchPoint}

\endmetadata

% this is the default PlanetMath preamble.  as your knowledge
% of TeX increases, you will probably want to edit this, but
% it should be fine as is for beginners.

% almost certainly you want these
\usepackage{amssymb}
\usepackage{amsmath}
\usepackage{amsfonts}

% used for TeXing text within eps files
%\usepackage{psfrag}
% need this for including graphics (\includegraphics)
%\usepackage{graphicx}
% for neatly defining theorems and propositions
 \usepackage{amsthm}
% making logically defined graphics
%%%\usepackage{xypic}
\usepackage{pstricks}
\usepackage{pst-plot}

% there are many more packages, add them here as you need them

% define commands here

\theoremstyle{definition}
\newtheorem*{thmplain}{Theorem}

\begin{document}
We take an example of applying the Cauchy residue theorem in evaluating usual real improper integrals.

We shall \PMlinkescapetext{calculate} the integral
$$I \;:=\; \int_0^\infty\frac{\cos{kx}}{1+x^2}\,dx$$
where $k$ is any real number.\, One may prove that the integrand has no antiderivative among the elementary functions if\, $k \neq 0$.

Since the integrand is an \PMlinkname{even}{EvenoddFunction} and\, $x\mapsto\frac{\sin{kx}}{1+x^2}$\, an odd function, we may write
$$I \;=\; \frac{1}{2}\int_{-\infty}^\infty\frac{\cos{kx}+i\sin{kx}}{1+x^2}\,dx
    \;=\; \frac{1}{2}\int_{-\infty}^\infty\frac{e^{ikx}}{1+x^2}\,dx,$$
using also Euler's formula.\, Let's consider the contour integral
$$J \;:=\; \oint_\gamma\frac{e^{ikz}}{1+z^2}\,dz$$
where $\gamma$ is the perimeter of the semicircle consisting of the line segment from $(-R,\,0)$ to $(R,\,0)$ and the semi-circular arc $c$ connecting these points in the upper half-plane ($R > 1$).\, The integrand is analytic on and inside of $\gamma$ except in the point\, $z = i$\, which is a simple pole.\, Because we have (cf. the coefficients of Laurent series)
$$\operatorname{Res}\left(\frac{e^{ikz}}{1+z^2},\;i\right) \;=\; 
  \lim_{z\to i}(z-i)\frac{e^{ikz}}{z^2+1} \;=\; \lim_{z\to i}\frac{e^{ikz}}{z+i}
\;=\; \frac{e^{-k}}{2i},$$
the \PMlinkname{residue theorem}{CauchyResidueTheorem} yields 
$$J \;=\; 2\pi i\!\cdot\!\frac{e^{-k}}{2i} \;=\; \pi e^{-k}.$$
This does not depend on the radius $R$ of the circle.
\begin{center}
\begin{pspicture}(-3,-1)(3,3)
\rput(-2.9,-1){.}
\rput(2.9,2.9){.}
\psline{->}(0,-0.7)(0,3)
\psline(-3,0)(-2.5,0)
\psline{->}(2.5,0)(3,0)
\psdot[linecolor=red](0,1)
\rput(-0.25,1){$i$}
\rput(2.5,-0.3){$R$}
\rput(-0.25,-0.25){0}
\rput(1.9,1.9){$c$}
\psline[linecolor=blue](-2.5,0)(2.5,0)
\psline[linecolor=blue](-2.5,0.02)(2.5,0.02)
\psarc[linecolor=blue](0,0){2.5}{0}{180}
\end{pspicture}
\end{center}
We split the integral $J$ in two portions:\, one along the diameter and the other along the circular arc $c$.\, So we obtain
$$\int_{-R}^R\frac{e^{ikx}}{1+x^2}\,dx+\!\int_c\frac{e^{ikz}}{1+z^2}\,dz \;\,=\;\, 
  \pi e^{-k}.$$
When \,$R\to\infty$,\, the former portion tends to the limit $2I$ and the latter --- as we at once shall see --- to the limit 0.\, Hence we get the result
$$I \;=\; \int_0^\infty\frac{\cos{kx}}{1+x^2}\,dx \;=\; \frac{\pi}{2e^k}.$$

As for the latter part of $J$, we denote\, $z := x+iy$ ($x,\,y\in\mathbb{R}$);\, then on the arc $c$, where\, $|z| = R$\, and \,$y\geqq 0$, we have
$$\left|\frac{e^{ikz}}{1+z^2}\right| \;=\; \frac{|e^{-ky+ikx}|}{|1+z^2|} 
 \;=\; \frac{e^{-ky}}{|1+z^2|} \;\leqq\; \frac{1}{R^2\!-\!1}.$$
Using this estimation of the integrand we get, according the integral estimating theorem, the inequality
$$\left|\int_c\frac{e^{ikz}}{1+z^2}\,dz\right| \;\leqq\; \frac{1}{R^2\!-\!1}\!\cdot\!\pi R \;=\; \frac{\pi}{R\!-\!\frac{1}{R}}.$$
Since the right hand member tends to 0 as\, $R\to\infty$, then also the left hand member.

%%%%%
%%%%%
\end{document}
