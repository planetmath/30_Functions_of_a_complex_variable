\documentclass[12pt]{article}
\usepackage{pmmeta}
\pmcanonicalname{QuasisymmetricMapping}
\pmcreated{2013-03-22 14:06:45}
\pmmodified{2013-03-22 14:06:45}
\pmowner{jirka}{4157}
\pmmodifier{jirka}{4157}
\pmtitle{quasisymmetric mapping}
\pmrecord{8}{35516}
\pmprivacy{1}
\pmauthor{jirka}{4157}
\pmtype{Definition}
\pmcomment{trigger rebuild}
\pmclassification{msc}{30C65}
\pmclassification{msc}{26A15}
\pmclassification{msc}{26A12}
\pmrelated{QuasiconformalMapping}
\pmrelated{BeurlingAhlforsQuasiconformalExtension}
\pmdefines{$M$-condition}
\pmdefines{quasisymmetric}
\pmdefines{$M$-quasisymmetric}

% this is the default PlanetMath preamble.  as your knowledge
% of TeX increases, you will probably want to edit this, but
% it should be fine as is for beginners.

% almost certainly you want these
\usepackage{amssymb}
\usepackage{amsmath}
\usepackage{amsfonts}

% used for TeXing text within eps files
%\usepackage{psfrag}
% need this for including graphics (\includegraphics)
%\usepackage{graphicx}
% for neatly defining theorems and propositions
%\usepackage{amsthm}
% making logically defined graphics
%%%\usepackage{xypic}

% there are many more packages, add them here as you need them

% define commands here
\begin{document}
A function $\mu$ of the real line to itself is \emph{quasisymmetric} (or \emph{$M$-quasisymmetric}) if it satisfies the following $M$-condition. 

There exists an $M$, such that for all $x, t$ (where $t \not= x$)
\begin{equation*}
\frac{1}{M}
\leq
\frac{\mu(x+t) - \mu(x)}{\mu(x)-\mu(x-t)}
\leq
M
.
\end{equation*}
Geometrically this means that the ratio of the length of the intervals $\mu[(x-t,x)]$ and $\mu[(x,x+t)]$ is bounded.  This implies among other things that the function is one-to-one and continuous.

For example powers (as long as you make them one-to-one by for example using an odd power, or defining them as $-|x|^p$ for negative $x$ and $|x|^p$ for positive $x$ where $p > 0$) are quasisymmetric.  On the other hand functions like $e^x - e^{-x}$, while one-to-one, onto and continuous, are not quasisymmetric.  It would seem like a very strict condition, however it has been shown that there in fact exist functions that are quasisymmetric, but are not even absolutely continuous.

Quasisymmetric functions are an analogue of quasiconformal mappings.
%%%%%
%%%%%
\end{document}
