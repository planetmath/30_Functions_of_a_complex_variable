\documentclass[12pt]{article}
\usepackage{pmmeta}
\pmcanonicalname{MartysTheorem}
\pmcreated{2013-03-22 14:18:39}
\pmmodified{2013-03-22 14:18:39}
\pmowner{jirka}{4157}
\pmmodifier{jirka}{4157}
\pmtitle{Marty's theorem}
\pmrecord{7}{35774}
\pmprivacy{1}
\pmauthor{jirka}{4157}
\pmtype{Theorem}
\pmcomment{trigger rebuild}
\pmclassification{msc}{30D30}

% this is the default PlanetMath preamble.  as your knowledge
% of TeX increases, you will probably want to edit this, but
% it should be fine as is for beginners.

% almost certainly you want these
\usepackage{amssymb}
\usepackage{amsmath}
\usepackage{amsfonts}

% used for TeXing text within eps files
%\usepackage{psfrag}
% need this for including graphics (\includegraphics)
%\usepackage{graphicx}
% for neatly defining theorems and propositions
\usepackage{amsthm}
% making logically defined graphics
%%%\usepackage{xypic}

% there are many more packages, add them here as you need them

% define commands here
\theoremstyle{theorem}
\newtheorem*{thm}{Theorem}
\newtheorem*{lemma}{Lemma}
\newtheorem*{conj}{Conjecture}
\newtheorem*{cor}{Corollary}
\newtheorem*{example}{Example}
\theoremstyle{definition}
\newtheorem*{defn}{Definition}
\begin{document}
\begin{thm}[Marty]
A set ${\mathcal{F}}$ of meromorphic functions is a normal family
on a domain $G \subset {\mathbb{C}}$ if and only if the spherical
derivatives are uniformly bounded (uniformly over ${\mathcal{F}}$)
on each compact subset of $G$.
\end{thm}

Here normal convergence (convergence on compact subsets) is given using the
spherical metric and not the standard metric of the complex plane.  That is, if
$\sigma$ is the spherical metric then we will say $f_n \to f$ normally
if $\sigma(f_n(z),f(z))$ converges to 0 uniformly on compact subsets.

A related theorem can be stated.

\begin{thm}
If $f_n(z) \to f(z)$ uniformly in the spherical metric on compact subsets of
$G \subset {\mathbb{C}}$ then $f_n^\sharp(z) \to f^\sharp(z)$ uniformly
on compact subsets of $G$.
\end{thm}

Here $f^\sharp$ denotes the spherical derivative of $f$.

\begin{thebibliography}{9}
\bibitem{Gamelin:complex}
Theodore~B.\@ Gamelin.
{\em \PMlinkescapetext{Complex Analysis}}.
Springer-Verlag, New York, New York, 2001.
\end{thebibliography}
%%%%%
%%%%%
\end{document}
