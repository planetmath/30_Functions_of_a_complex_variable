\documentclass[12pt]{article}
\usepackage{pmmeta}
\pmcanonicalname{IncompleteGammaFunctionRecurrenceFormula}
\pmcreated{2013-03-22 15:36:50}
\pmmodified{2013-03-22 15:36:50}
\pmowner{rspuzio}{6075}
\pmmodifier{rspuzio}{6075}
\pmtitle{incomplete gamma function recurrence formula}
\pmrecord{6}{37537}
\pmprivacy{1}
\pmauthor{rspuzio}{6075}
\pmtype{Theorem}
\pmcomment{trigger rebuild}
\pmclassification{msc}{30D30}
\pmclassification{msc}{33B15}

\endmetadata

% this is the default PlanetMath preamble.  as your knowledge
% of TeX increases, you will probably want to edit this, but
% it should be fine as is for beginners.

% almost certainly you want these
\usepackage{amssymb}
\usepackage{amsmath}
\usepackage{amsfonts}

% used for TeXing text within eps files
%\usepackage{psfrag}
% need this for including graphics (\includegraphics)
%\usepackage{graphicx}
% for neatly defining theorems and propositions
%\usepackage{amsthm}
% making logically defined graphics
%%%\usepackage{xypic}

% there are many more packages, add them here as you need them

% define commands here
\begin{document}
The incomplete gamma function satisfies the following recurrence formula:

\[ \gamma (a+1,x) = a \gamma (a,x) - x^a e^{-x} \]

This can be derived by integration by parts:

\begin{eqnarray*}
\int_0^x e^{-t} t^a \, dt &=&
- \int_0^x t^a \, d e^{-t} \\
&=& a \int_0^x t^{a-1} \, d e^{-t} - x^a e^{-x}
\end{eqnarray*}

In terms of other variants of the incomplete gamma function, the recursion relation appears as follows:

\begin{eqnarray*}
P (a+1,x) &=& P (a,x) - {x^a e^{-x} \over \Gamma (a+1)} \\
\gamma^* (a-1,x) &=& x \gamma^* (a,x) + {e^{-x} \over \Gamma(a)} \\
\end{eqnarray*}
%%%%%
%%%%%
\end{document}
