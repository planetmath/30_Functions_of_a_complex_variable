\documentclass[12pt]{article}
\usepackage{pmmeta}
\pmcanonicalname{Koebe14Theorem}
\pmcreated{2013-03-22 14:23:57}
\pmmodified{2013-03-22 14:23:57}
\pmowner{jirka}{4157}
\pmmodifier{jirka}{4157}
\pmtitle{Koebe 1/4 theorem}
\pmrecord{8}{35896}
\pmprivacy{1}
\pmauthor{jirka}{4157}
\pmtype{Theorem}
\pmcomment{trigger rebuild}
\pmclassification{msc}{30C45}
\pmsynonym{K\"obe 1/4 theorem}{Koebe14Theorem}
\pmsynonym{Koebe one-fourth theorem}{Koebe14Theorem}
\pmsynonym{K\"obe one-fourth theorem}{Koebe14Theorem}
\pmrelated{SchlichtFunctions}

% this is the default PlanetMath preamble.  as your knowledge
% of TeX increases, you will probably want to edit this, but
% it should be fine as is for beginners.

% almost certainly you want these
\usepackage{amssymb}
\usepackage{amsmath}
\usepackage{amsfonts}

% used for TeXing text within eps files
%\usepackage{psfrag}
% need this for including graphics (\includegraphics)
%\usepackage{graphicx}
% for neatly defining theorems and propositions
\usepackage{amsthm}
% making logically defined graphics
%%%\usepackage{xypic}

% there are many more packages, add them here as you need them

% define commands here
\theoremstyle{theorem}
\newtheorem*{thm}{Theorem}
\newtheorem*{lemma}{Lemma}
\newtheorem*{conj}{Conjecture}
\newtheorem*{cor}{Corollary}
\newtheorem*{example}{Example}
\theoremstyle{definition}
\newtheorem*{defn}{Definition}
\begin{document}
\begin{thm}[Koebe]
Suppose $f$ is a schlicht function (univalent function on the unit disc
such that $f(0) = 0$ and $f'(0) = 1$) and ${\mathbb{D}} \subset {\mathbb{C}}$ is the unit disc in the complex plane, then
\begin{equation*}
f({\mathbb{D}}) \supseteq \{ w \mid \lvert w \rvert < 1/4 \} .
\end{equation*}
\end{thm}

That is, if a univalent function on the unit disc maps 0 to 0 and has derivative 
1 at 0, then the image of the unit disc contains the ball of radius $1/4$.  So for any $w \notin f({\mathbb{D}})$ we have that $\lvert w \rvert \geq 1/4$.  Furthermore, if we look at the Koebe function, we can see that the constant $1/4$ is sharp and cannot be improved.

\begin{thebibliography}{9}
\bibitem{Conway:complexII}
John~B. Conway.
{\em \PMlinkescapetext{Functions of One Complex Variable II}}.
Springer-Verlag, New York, New York, 1995.
\end{thebibliography}
%%%%%
%%%%%
\end{document}
