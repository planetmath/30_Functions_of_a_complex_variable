\documentclass[12pt]{article}
\usepackage{pmmeta}
\pmcanonicalname{ExampleOfAnalyticContinuation}
\pmcreated{2013-03-22 16:52:06}
\pmmodified{2013-03-22 16:52:06}
\pmowner{pahio}{2872}
\pmmodifier{pahio}{2872}
\pmtitle{example of analytic continuation}
\pmrecord{5}{39118}
\pmprivacy{1}
\pmauthor{pahio}{2872}
\pmtype{Example}
\pmcomment{trigger rebuild}
\pmclassification{msc}{30B40}
\pmclassification{msc}{30A99}
\pmrelated{RadiusOfConvergence}
\pmrelated{GeometricSeries}
\pmrelated{SetDifference}

% this is the default PlanetMath preamble.  as your knowledge
% of TeX increases, you will probably want to edit this, but
% it should be fine as is for beginners.

% almost certainly you want these
\usepackage{amssymb}
\usepackage{amsmath}
\usepackage{amsfonts}

% used for TeXing text within eps files
%\usepackage{psfrag}
% need this for including graphics (\includegraphics)
%\usepackage{graphicx}
% for neatly defining theorems and propositions
 \usepackage{amsthm}
% making logically defined graphics
%%%\usepackage{xypic}

% there are many more packages, add them here as you need them

% define commands here

\theoremstyle{definition}
\newtheorem*{thmplain}{Theorem}

\begin{document}
The function defined by
        $$f(z) := \sum_{n=0}^\infty z^n = 1+z+z^2+\ldots$$
is, as a sum of power series, analytic in the disc of convergence\, 
$D = \{z\in\mathbb{C}\,\vdots\;\; |z| < 1\}$.\, The function
   $$g(z) := \frac{1}{1-i}\sum_{n=0}^\infty\left(\frac{z-i}{1-i}\right)^n
= \frac{1}{1-i}+\frac{z-i}{(1-i)^2}+\frac{(z-i)^2}{(1-i)^3}+\ldots$$
similarly is analytic in the bigger disc\, 
$E = \{z\in\mathbb{C}\,\vdots\;\; |z-i| < \sqrt{2}\}$.\, But we have
$$f(z) = \frac{1}{1-z}\quad\mathrm{in}\; D,$$
$$g(z) = \frac{1}{1-i}\cdot\frac{1}{1-\frac{z-i}{1-i}} = \frac{1}{1-z}
\quad\mathrm{in}\; E;$$
thus $f(z)$ and $g(z)$ coincide in the intersection domain $D \cap E$.\, So we can say that $g(z)$ is the analytic continuation of $f(z)$ to the domain\, $E\!\smallsetminus\!D$.\, It is clear that $\frac{1}{1-z}$ is the analytic continuation of $f(z)$ to the domain $\mathbb{C}\!\smallsetminus\!\{1\}$.

%%%%%
%%%%%
\end{document}
