\documentclass[12pt]{article}
\usepackage{pmmeta}
\pmcanonicalname{MonodromyGroup}
\pmcreated{2013-03-22 16:21:51}
\pmmodified{2013-03-22 16:21:51}
\pmowner{rspuzio}{6075}
\pmmodifier{rspuzio}{6075}
\pmtitle{monodromy group}
\pmrecord{11}{38501}
\pmprivacy{1}
\pmauthor{rspuzio}{6075}
\pmtype{Definition}
\pmcomment{trigger rebuild}
\pmclassification{msc}{30A99}
\pmrelated{Monodromy}

% this is the default PlanetMath preamble.  as your knowledge
% of TeX increases, you will probably want to edit this, but
% it should be fine as is for beginners.

% almost certainly you want these
\usepackage{amssymb}
\usepackage{amsmath}
\usepackage{amsfonts}

% used for TeXing text within eps files
%\usepackage{psfrag}
% need this for including graphics (\includegraphics)
%\usepackage{graphicx}
% for neatly defining theorems and propositions
%\usepackage{amsthm}
% making logically defined graphics
%%%\usepackage{xypic}

% there are many more packages, add them here as you need them

% define commands here

\begin{document}
Consider an ordinary linear differential equation 
 \[\sum_{k = 0}^n c_k (x) {d^n y \over dx^n} = 0\]
in which the coefficients are polynomials.  If $c_n$ 
is not constant, then it is possible that the solutions
of this equation will have branch points at the zeros
of $c_n$.  (To see if this actually happens, we need to
examine the indicial equation.)

By the persistence of differential equations, the analytic
continuation of a solution of this equation will be another
solution.  Pick a neighborhood which does not contain any
zeros of $c_n$.  Since the differential equation is of order 
$n$, there will be $n$ independent solutions $y = f_1 (x),
\ldots, y = f_n (x)$.  (For example, one may exhibit these
solutions as power series about some point in the neighborhood.)

Upon analytic continuation back to the original neighborhood via 
a chain of neghborhoods, suppose 
that the solution $y = f_i (x)$ is taken to a solution $y = {g}_i (x)$.
Because the solutions are linearly independent, there will exist a matrix 
$\{m_{ij}\}_{i,j=1}^n$ such that
 \[ {g}_i = \sum_{j=1}^n m_{ij} f_j. \]
Now consider the totality of all such matrices corresponding to
all possible ways of making analytic continuations along chains
which begin and end wit the original neighborhood.  They form a
group known as the \emph{monodromy group} of the differential 
equation.  The reason this set is a group is some basic facts
about analytic continuation.  First, there is the trivial analytic
continuation which simply takes a function element to itself.  This 
will correspond to the identity matrix.  Second, we can reverse a
process of analytic continuation.  This will correspond to the inverse
matrix.  Third, we can follow continuation along one chain of 
neighborhoods by continuation along another chain.  This will 
correspond to multiplying the matrices corresponding to the two chains.


%%%%%
%%%%%
\end{document}
