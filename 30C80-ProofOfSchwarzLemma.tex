\documentclass[12pt]{article}
\usepackage{pmmeta}
\pmcanonicalname{ProofOfSchwarzLemma}
\pmcreated{2013-03-22 12:45:07}
\pmmodified{2013-03-22 12:45:07}
\pmowner{Mathprof}{13753}
\pmmodifier{Mathprof}{13753}
\pmtitle{proof of Schwarz lemma}
\pmrecord{6}{33057}
\pmprivacy{1}
\pmauthor{Mathprof}{13753}
\pmtype{Proof}
\pmcomment{trigger rebuild}
\pmclassification{msc}{30C80}

\endmetadata

% this is the default PlanetMath preamble.  as your knowledge
% of TeX increases, you will probably want to edit this, but
% it should be fine as is for beginners.

% almost certainly you want these
\usepackage{amssymb}
\usepackage{amsmath}
\usepackage{amsfonts}

% used for TeXing text within eps files
%\usepackage{psfrag}
% need this for including graphics (\includegraphics)
%\usepackage{graphicx}
% for neatly defining theorems and propositions
%\usepackage{amsthm}
% making logically defined graphics
%%%\usepackage{xypic}

% there are many more packages, add them here as you need them

% define commands here

\newcommand{\Prob}[2]{\mathbb{P}_{#1}\left\{#2\right\}}
\newcommand{\norm}[1]{\left\|#1\right\|}

% Some sets
\newcommand{\Nats}{\mathbb{N}}
\newcommand{\Ints}{\mathbb{Z}}
\newcommand{\Reals}{\mathbb{R}}
\newcommand{\Complex}{\mathbb{C}}


\newcommand{\size}[1]{\left|#1\right|}
\begin{document}
Define $g(z)=f(z)/z$.  Then $g:\Delta\to\Complex$ is a holomorphic function.  The Schwarz lemma is just an application of the maximal modulus principle to $g$.

For any $1>\epsilon>0$, by the maximal modulus principle $\size{g}$ must attain its maximum on the closed disk $\left\{z:\size{z}\le 1-\epsilon\right\}$ at its boundary $\left\{z:\size{z}=1-\epsilon\right\}$, say at some point $z_{\epsilon}$. But then $\size{g(z)}\le \size{g(z_{\epsilon})} \le \frac{1}{1-\epsilon}$ for any $\size{z}\le 1-\epsilon$.  Taking an infinimum as $\epsilon\to0$, we see that values of $g$ are bounded: $\size{g(z)}\le 1$.

Thus $\size{f(z)}\le\size{z}$.  Additionally, $f'(0)=g(0)$, so we see that $\size{f'(0)}=\size{g(0)}\le 1$.  This is the first part of the lemma.

Now suppose, as per the premise of the second part of the lemma, that $|g(w)|=1$ for some $w\in\Delta$.  For any $r>\size{w}$, it must be that $\size{g}$ attains its maximal modulus (1) \emph{inside} the disk $\left\{z:\size{z}\le r\right\}$, and it follows that $g$ must be constant inside the entire open disk $\Delta$.  So $g(z)\equiv a$ for $a=g(w)$ of modulus 1, and $f(z)=az$, as required.
%%%%%
%%%%%
\end{document}
