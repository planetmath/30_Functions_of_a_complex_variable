\documentclass[12pt]{article}
\usepackage{pmmeta}
\pmcanonicalname{ConformalMapping}
\pmcreated{2013-03-22 13:35:42}
\pmmodified{2013-03-22 13:35:42}
\pmowner{rspuzio}{6075}
\pmmodifier{rspuzio}{6075}
\pmtitle{conformal mapping}
\pmrecord{19}{34219}
\pmprivacy{1}
\pmauthor{rspuzio}{6075}
\pmtype{Definition}
\pmcomment{trigger rebuild}
\pmclassification{msc}{30E20}
\pmclassification{msc}{53A30}
\pmrelated{QuasiconformalMapping}
\pmrelated{ConformalMappingTheorem}
\pmrelated{AngleBetweenTwoLines}
\pmrelated{SchwarzChristoffelTransformation}
\pmrelated{CategoryOfRiemannianManifolds}
\pmdefines{conformal}
\pmdefines{inversely conformal}

\endmetadata

% this is the default PlanetMath preamble.  as your knowledge
% of TeX increases, you will probably want to edit this, but
% it should be fine as is for beginners.

% almost certainly you want these
\usepackage{amssymb}
\usepackage{amsmath}
\usepackage{amsfonts}

% used for TeXing text within eps files
%\usepackage{psfrag}
% need this for including graphics (\includegraphics)
%\usepackage{graphicx}
% for neatly defining theorems and propositions
%\usepackage{amsthm}
% making logically defined graphics
%%%\usepackage{xypic}

% there are many more packages, add them here as you need them

% define commands here
\begin{document}
A mapping $f \colon \mathbb{R}^m \to \mathbb{R}^n$ which preserves the magnitude and orientation of the angles between any two curves which intersect in a given point $z_0$ is said to be conformal at $z_0$. A mapping that is conformal at any point in a domain $D$ is said to be conformal in $D$.

An important special case is when $m = n = 2$.  In this case, we may identify $\mathbb{R}^2$ with $\mathbb{C}$ and speak of conformal mappings of the complex plane.  It can be shown that a mapping $f \colon \mathbb{C} \to \mathbb{C}$ is conformal if and only if $f$ is a complex analytic function.\, The complex conjugate $\overline{f}$ of a conformal mapping $f$ is {\em inversely conformal}, i.e. it preserves the magnitude but reverses the orientation of angles.

If $m = n$, then we can study invertible conformal mappings.  It is clear from the definition that the composition of two such maps and the inverse of any such map is again an invertible conformal mapping, so the set of such mappings forms a group.  In the case $m = n > 2$, the group of conformal mappings is finite dimensional and is generated by rotations, translations, and spherical inversions.

This notion of conformal mappings can be generalized to any setting in which it makes sense to speak of angles between curves or angles between tangent vectors.  In particular, one can consider conformal mappings of Riemannian manifolds.  It can be shown that, if $(M,g)$ and $(N,h)$ are Riemannian manifolds, then a map $f \colon M \to N$ is conformal if and only if $f^* h = s g$ for some scalar field $s$ (on $M$).
%%%%%
%%%%%
\end{document}
