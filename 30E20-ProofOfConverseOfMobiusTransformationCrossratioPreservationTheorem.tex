\documentclass[12pt]{article}
\usepackage{pmmeta}
\pmcanonicalname{ProofOfConverseOfMobiusTransformationCrossratioPreservationTheorem}
\pmcreated{2013-03-22 17:01:51}
\pmmodified{2013-03-22 17:01:51}
\pmowner{rspuzio}{6075}
\pmmodifier{rspuzio}{6075}
\pmtitle{proof of converse of M\"obius transformation cross-ratio preservation theorem}
\pmrecord{6}{39316}
\pmprivacy{1}
\pmauthor{rspuzio}{6075}
\pmtype{Proof}
\pmcomment{trigger rebuild}
\pmclassification{msc}{30E20}

\endmetadata

% this is the default PlanetMath preamble.  as your knowledge
% of TeX increases, you will probably want to edit this, but
% it should be fine as is for beginners.

% almost certainly you want these
\usepackage{amssymb}
\usepackage{amsmath}
\usepackage{amsfonts}

% used for TeXing text within eps files
%\usepackage{psfrag}
% need this for including graphics (\includegraphics)
%\usepackage{graphicx}
% for neatly defining theorems and propositions
%\usepackage{amsthm}
% making logically defined graphics
%%%\usepackage{xypic}

% there are many more packages, add them here as you need them

% define commands here

\begin{document}
Suppose that $a,b,c,d$ are distinct.  Consider the transform $\mu$ defined as
\[
\mu (z) = {(b-d) (c-d) \over (c-b) (z-d)} - {b-d \over c-b} .
\]
Simple calculation reveals that $\mu (b) = 1$, $\mu (c) = 0$, and $\mu(d) = \infty$.
Furthermore, $\mu (a)$ equals the cross-ratio of $a,b,c,d$.

Suppose we have two tetrads with a common cross-ratio $\lambda$.  Then, as above, we may
construct a transform $\mu_1$ which maps the first tetrad to $(\lambda, 1, 0, \infty)$ and a 
transform $\mu_2$ which maps the first tetrad to $(\lambda, 1, 0, \infty)$.  Then $\mu_2^{-1} 
\circ \mu_1$ maps the former tetrad to the latter and, by the group property, it is also a
M\"obius transformation.

%%%%%
%%%%%
\end{document}
