\documentclass[12pt]{article}
\usepackage{pmmeta}
\pmcanonicalname{IdentityTheorem}
\pmcreated{2013-03-22 17:10:38}
\pmmodified{2013-03-22 17:10:38}
\pmowner{fernsanz}{8869}
\pmmodifier{fernsanz}{8869}
\pmtitle{identity theorem}
\pmrecord{8}{39491}
\pmprivacy{1}
\pmauthor{fernsanz}{8869}
\pmtype{Theorem}
\pmcomment{trigger rebuild}
\pmclassification{msc}{30E99}
%\pmkeywords{zeros}
%\pmkeywords{poles}
%\pmkeywords{analytic functions}
%\pmkeywords{indentity}
%\pmkeywords{complex plane}
%\pmkeywords{clopen}
\pmrelated{Complex}
\pmrelated{ZeroesOfAnalyticFunctionsAreIsolated}
\pmrelated{TopologyOfTheComplexPlane}
\pmrelated{ClopenSubset}
\pmrelated{IdentityTheoremOfHolomorphicFunctions}
\pmrelated{PlacesOfHolomorphicFunction}

% this is the default PlanetMath preamble.  as your knowledge
% of TeX increases, you will probably want to edit this, but
% it should be fine as is for beginners.

% almost certainly you want these
\usepackage{amssymb}
\usepackage{amsmath}
\usepackage{amsfonts}
\usepackage{amsthm}

% used for TeXing text within eps files
%\usepackage{psfrag}
% need this for including graphics (\includegraphics)
%\usepackage{graphicx}

% define commands here
\newtheorem{thm}{Theorem}
\newtheorem{cor}[thm]{Corollary}
\newtheorem{lem}{Lemma}
\theoremstyle{remark}
\newtheorem{rem}{Remark}
\numberwithin{equation}{section}
\begin{document}
\title{Identity theorem}%
\author{Fernando Sanz Gamiz}%

\begin{lem}
Let $f$ be analytic on $\Omega \subseteq \mathbb C$ and let $L$ be the set of
accumulation points (also called limit points or cluster points) of
$\{z \in \Omega \colon f(z)=0\}$ in $\Omega$. Then $L$ is both open
and closed in $\Omega$.
\end{lem}

\bigskip

\begin{proof}
By definition of accumulation point, $L$ is closed. To see that it
is also open, let $z_0 \in L$, choose an open ball $B(z_0, r)
\subseteq \Omega$ and write $f(z)=\sum_{n=0}^{\infty} a_n(z-z_0)^n,
z \in B(z_0, r)$. Now $f(z_0) = 0$, and hence either $f$ has a zero
of order $m$ at $z_0$ (for some $m$), or else $a_n = 0$ for all $n$.
In the former case, there is a function g analytic on $\Omega$  such
that $f(z)= (z-z_0)^m g(z), z \in \Omega$, with $g(z_0) \neq 0$. By
continuity of $g$, $g(z) \neq 0$ for all $z$ sufficiently close to
$z_0$, and consequently $z_0$ is an isolated point of$\{z \in \Omega
\colon f(z)=0\}$ . But then $z_0 \notin L$, contradicting our
assumption. Thus, it must be the case that $a_n = 0$ for all n, so
that $f \equiv 0$ on $B(z_0, r)$. Consequently, $B(z_0, r) \in L$,
proving that $L$ is open in $\Omega$.
\end{proof}

\bigskip

\begin{thm}[Identity theorem]
Let $\Omega$ be a open connected subset of $\mathbb C$ (i.e., a domain). If $f$ and
$g$ are analytic on $\Omega$ and $\{z \in \Omega \colon f(z)=g(z)\}$
has an accumulation point in $\Omega$, then $f \equiv g$ on
$\Omega$.
\end{thm}

\bigskip

\begin{proof}
We have that $\{z \in \Omega \colon f(z)-g(z)=0\}$ has an
accumulation point, hence, according to the previous lemma, it is
open and closed (also called "clopen"). But, as $\Omega$ is
connected, the only closed and open subset at once is $\Omega$
itself, therefore $\{z \in \Omega \colon f(z)-g(z)=0\}=\Omega$,
i.e., $f \equiv g$ on $\Omega$.
\end{proof}

\bigskip

\begin{rem}
This theorem provides a very powerful and useful tool to test
whether two analytic functions, whose values coincide in some
points, are indeed the same function. Namely, unless the points in
which they are equal are isolated, they are the same function.
\end{rem}

%%%%%
%%%%%
\end{document}
