\documentclass[12pt]{article}
\usepackage{pmmeta}
\pmcanonicalname{AdditionFormulas}
\pmcreated{2013-03-22 19:35:28}
\pmmodified{2013-03-22 19:35:28}
\pmowner{pahio}{2872}
\pmmodifier{pahio}{2872}
\pmtitle{addition formulas}
\pmrecord{6}{42580}
\pmprivacy{1}
\pmauthor{pahio}{2872}
\pmtype{Definition}
\pmcomment{trigger rebuild}
\pmclassification{msc}{30D05}
\pmclassification{msc}{30A99}
\pmclassification{msc}{26A99}
\pmrelated{ExampleOnSolvingAFunctionalEquation}
\pmrelated{ProofOfAdditionFormulaOfExp}
\pmrelated{AdditionFormulasForSineAndCosine}
\pmrelated{AdditionFormulaForTangent}
\pmrelated{AdditionAndSubtractionFormulasForHyperbolicFunctions}
\pmdefines{addition formula}
\pmdefines{subtraction formula}

% this is the default PlanetMath preamble.  as your knowledge
% of TeX increases, you will probably want to edit this, but
% it should be fine as is for beginners.

% almost certainly you want these
\usepackage{amssymb}
\usepackage{amsmath}
\usepackage{amsfonts}

% used for TeXing text within eps files
%\usepackage{psfrag}
% need this for including graphics (\includegraphics)
%\usepackage{graphicx}
% for neatly defining theorems and propositions
 \usepackage{amsthm}
% making logically defined graphics
%%%\usepackage{xypic}

% there are many more packages, add them here as you need them

% define commands here

\theoremstyle{definition}
\newtheorem*{thmplain}{Theorem}

\begin{document}
\PMlinkescapeword{connection}

The {\em addition formula} of a \PMlinkname{real}{RealFunction} or complex function shows how the value of the function at a sum-formed variable can be expressed with the values of this function and perhaps of another function at the addends.\\

\textbf{Examples}
\begin{enumerate}

\item Addition formula of an additive function $f$,\\
$f(x\!+\!y) = f(x)+f(y)$

\item Addition formula of the natural power function, i.e. the binomial theorem,\\
$(x\!+\!y)^n = \sum_{\nu = 0}^n {n\choose \nu} x^{\nu}y^{n-\nu}\qquad(n = 0,\,1,\,2,\,\ldots)$

\item Addition formula of the \PMlinkname{exponential function}{ComplexExponentialFunction},\\
$e^{x+y} = e^xe^y$

\item Addition formulae of the \PMlinkname{trigonometric functions}{DefinitionsInTrigonometry}, e.g.\\
$\cos(x\!+\!y) = \cos{x}\cos{y}-\sin{x}\sin{y},\footnote{The addition formula of cosine is sometimes called ``the mother of all formulae''.}\,\,\,\,
\tan(x\!+\!y) = \frac{\tan{x}+\tan{y}}{1-\tan{x}\tan{y}}$

\item Addition formulae of the hyperbolic functions, e.g.\\
$\sinh(x\!+\!y) = \sinh{x}\cosh{y}+\cosh{x}\sinh{y}$

\item Addition formula of the Bessel function,\\
$J_n(x\!+\!y) = \sum_{\nu=-\infty}^{\infty}J_\nu(x)J_{n-\nu}(y)
\qquad(n = 0,\,\pm1,\,\pm2,\,\ldots)$

\end{enumerate}

The five first of those are instances of \PMlinkescapetext{{\em algebraic addition formulae}}; e.g. $\cosh{x}$\, and \,$\sinh{x}$\, are tied together by the algebraic \PMlinkname{connection}{UnitHyperbola} \,$\cosh^2{x}-\sinh^2{x} = 1$.

One may also speak of the {\em subtraction formulae} of functions --- one example would be\, $e^{x-y} = \frac{e^x}{e^y}$.

%%%%%
%%%%%
\end{document}
