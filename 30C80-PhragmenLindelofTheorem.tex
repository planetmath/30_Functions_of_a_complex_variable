\documentclass[12pt]{article}
\usepackage{pmmeta}
\pmcanonicalname{PhragmenLindelofTheorem}
\pmcreated{2013-03-22 14:12:09}
\pmmodified{2013-03-22 14:12:09}
\pmowner{jirka}{4157}
\pmmodifier{jirka}{4157}
\pmtitle{Phragm\'en-Lindel\"of theorem}
\pmrecord{9}{35634}
\pmprivacy{1}
\pmauthor{jirka}{4157}
\pmtype{Theorem}
\pmcomment{trigger rebuild}
\pmclassification{msc}{30C80}
\pmsynonym{Phragm\'en-Lindel\"of principle}{PhragmenLindelofTheorem}
\pmrelated{MaximalModulusPrinciple}

\endmetadata

% this is the default PlanetMath preamble.  as your knowledge
% of TeX increases, you will probably want to edit this, but
% it should be fine as is for beginners.

% almost certainly you want these
\usepackage{amssymb}
\usepackage{amsmath}
\usepackage{amsfonts}

% used for TeXing text within eps files
%\usepackage{psfrag}
% need this for including graphics (\includegraphics)
%\usepackage{graphicx}
% for neatly defining theorems and propositions
\usepackage{amsthm}
% making logically defined graphics
%%%\usepackage{xypic}

% there are many more packages, add them here as you need them

% define commands here
\theoremstyle{theorem}
\newtheorem*{thm}{Theorem}
\newtheorem*{lemma}{Lemma}
\newtheorem*{conj}{Conjecture}
\newtheorem*{cor}{Corollary}
\newtheorem*{example}{Example}
\theoremstyle{definition}
\newtheorem*{defn}{Definition}
\begin{document}
First some notation.  Let $\partial_\infty G$ be the extended boundary of $G$.  That is, the boundary of $G$, plus optionally the point at infinity if in fact
$G$ is unbounded.

\begin{thm}
Let $G$ be a simply connected region and let $f \colon G \to {\mathbb{C}}$
and $\varphi \colon G \to {\mathbb{C}}$ be analytic functions.  Furthermore
suppose that $\varphi$ never vanishes and is bounded on $G$.  If $M$ is a constant
and $\partial_\infty G = A \cup B$ such that
\begin{enumerate}
\item for every $a \in A$, $\varlimsup_{z\rightarrow a} \lvert f(z) \rvert \leq M$, and
\item for every $b \in B$, and $\eta > 0$, $\varlimsup_{z \rightarrow b} \lvert f(z) \rvert \lvert \varphi(z) \rvert^\eta \leq M$,
\end{enumerate}
then $\lvert f(z) \rvert \leq M$ for all $z \in G$.
\end{thm}

This theorem is a generalization of the maximal modulus principle, but instead of requiring that the function is bounded as we approach the boundary, we only need a restriction on its growth to \PMlinkescapetext{force} it to in fact be bounded in all of $G$.

If you let $A = \partial_\infty G$ (and $\varphi \equiv 1$ perhaps), then you get almost exactly one version of the maximal modulus principle.  In this case it turns out that $G$ need not be simply connected since that is only needed to
define $z \mapsto \varphi(z)^\eta$.

In fact the requirement that $G$ be simply connected can be eased up a bit in this theorem since it is only needed locally.  So the theorem is still true if for every point
$z \in \partial_\infty G$ there exists an open neighbourhood $N$ of $z$ such
that $N \cap G$ is simply connected.

\begin{thebibliography}{9}
\bibitem{Conway:complexI}
John~B. Conway.
{\em \PMlinkescapetext{Functions of One Complex Variable I}}.
Springer-Verlag, New York, New York, 1978.
\end{thebibliography}
%%%%%
%%%%%
\end{document}
