\documentclass[12pt]{article}
\usepackage{pmmeta}
\pmcanonicalname{ExampleOfConformalMapping}
\pmcreated{2013-03-22 13:36:36}
\pmmodified{2013-03-22 13:36:36}
\pmowner{Johan}{1032}
\pmmodifier{Johan}{1032}
\pmtitle{example of conformal mapping}
\pmrecord{6}{34241}
\pmprivacy{1}
\pmauthor{Johan}{1032}
\pmtype{Example}
\pmcomment{trigger rebuild}
\pmclassification{msc}{30E20}
\pmrelated{CategoryOfRiemannianManifolds}

\endmetadata

% this is the default PlanetMath preamble.  as your knowledge
% of TeX increases, you will probably want to edit this, but
% it should be fine as is for beginners.

% almost certainly you want these
\usepackage{amssymb}
\usepackage{amsmath}
\usepackage{amsfonts}

% used for TeXing text within eps files
%\usepackage{psfrag}
% need this for including graphics (\includegraphics)
%\usepackage{graphicx}
% for neatly defining theorems and propositions
%\usepackage{amsthm}
% making logically defined graphics
%%%\usepackage{xypic}

% there are many more packages, add them here as you need them

% define commands here
\begin{document}
Consider the four curves $A=\{t\}$, $B=\{t+it\}$, $C=\{it\}$ and $D=\{-t+it\}$, $t\in[-10,10]$. Suppose there is a mapping $f:\mathbb{C}\mapsto\mathbb{C}$ which maps $A$ to $D$ and $B$ to $C$. Is $f$ conformal at $z_0=0$? The size of the angles between $A$ and $B$ at the point of intersection $z_0=0$ is preserved, however the orientation is not. Therefore $f$ is not conformal at $z_0=0$. Now suppose there is a function $g:\mathbb{C}\mapsto\mathbb{C}$ which maps $A$ to $C$ and $B$ to $D$. In this case we see not only that the size of the angles is preserved, but also the orientation. Therefore $g$ is conformal at $z_0=0$.
%%%%%
%%%%%
\end{document}
