\documentclass[12pt]{article}
\usepackage{pmmeta}
\pmcanonicalname{ProofOfCauchyResidueTheorem}
\pmcreated{2013-03-22 13:42:04}
\pmmodified{2013-03-22 13:42:04}
\pmowner{paolini}{1187}
\pmmodifier{paolini}{1187}
\pmtitle{proof of Cauchy residue theorem}
\pmrecord{7}{34377}
\pmprivacy{1}
\pmauthor{paolini}{1187}
\pmtype{Proof}
\pmcomment{trigger rebuild}
\pmclassification{msc}{30E20}

% this is the default PlanetMath preamble.  as your knowledge
% of TeX increases, you will probably want to edit this, but
% it should be fine as is for beginners.

% almost certainly you want these
\usepackage{amssymb}
\usepackage{amsmath}
\usepackage{amsfonts}

% used for TeXing text within eps files
%\usepackage{psfrag}
% need this for including graphics (\includegraphics)
%\usepackage{graphicx}
% for neatly defining theorems and propositions
%\usepackage{amsthm}
% making logically defined graphics
%%%\usepackage{xypic}

% there are many more packages, add them here as you need them

% define commands here
\begin{document}
Being $f$ holomorphic by Cauchy-Riemann equations the differential form $f(z)\,dz$ is closed. So by the lemma about closed differential forms on a simple connected domain we know that the integral $\int_C f(z)\, dz$ is equal to $\int_{C'} f(z)\, dz$ if $C'$ is any curve which is homotopic to $C$. 
In particular we can consider a curve $C'$ which turns around the points $a_j$ along small circles and join these small circles with segments. Since the curve $C'$ follows each segment two times with opposite orientation it is enough to sum
the integrals of $f$ around the small circles.

So letting $z=a_j+\rho e^{i\theta}$ be a parameterization of the curve around the point $a_j$, we have $dz=\rho i e^{i\theta}\, d \theta$ and hence
\[
  \int_C f(z)\, dz = \int_{C'} f(z)\, dz = \sum_j \eta(C,a_j)\int_{\partial B_\rho(a_j)} f(z)\, dz
\]\[
 = \sum_j \eta(C,a_j) \int_0^{2\pi} f(a_j+\rho e^{i\theta}) \rho i e^{i\theta}\, d\theta
\]
where $\rho>0$ is chosen so small that the balls $B_\rho(a_j)$ are all disjoint and all contained in the domain $U$. So by linearity, it is enough to prove that
for all $j$
\[
   i\int_0^{2\pi} f(a_j+e^{i\theta})\rho e^{i\theta}\, d\theta = 2\pi i \mathrm{Res}(f,a_j).
\]

Let now $j$ be fixed and
consider now the Laurent series for $f$ in $a_j$:
\[
f(z)= \sum_{k\in \mathbb Z} c_k (z-a_j)^k
\]
so that $\mathrm{Res}(f,a_j)=c_{-1}$. We have
\[
  \int_0^{2\pi} f(a_j+e^{i\theta})\rho e^{i\theta}\, d\theta =
  \sum_k \int_0^{2\pi} c_k (\rho e^{i\theta})^k \rho e^{i\theta}\, d\theta
 =\rho^{k+1} \sum_k c_k \int_0^{2\pi}  e^{i(k+1)\theta}\, d\theta.
\]
Notice now that if $k=-1$ we have
\[
\rho^{k+1} c_k \int_0^{2\pi} e^{i(k+1)\theta}\, d\theta = 
c_{-1}\int_0^{2\pi} d\theta = 2\pi c_{-1} = 2\pi \,\mathrm{Res}(f,a_j)
\]
while for $k\neq -1$ we have
\[
  \int_0^{2\pi} e^{i(k+1)\theta}\, d\theta =  \left[\frac{e^{i(k+1)\theta}}{i(k+1)}\right]_0^{2\pi} = 0.
\]
Hence the result follows.
%%%%%
%%%%%
\end{document}
