\documentclass[12pt]{article}
\usepackage{pmmeta}
\pmcanonicalname{ProofOfClosedCurveTheorem}
\pmcreated{2013-03-22 13:33:34}
\pmmodified{2013-03-22 13:33:34}
\pmowner{paolini}{1187}
\pmmodifier{paolini}{1187}
\pmtitle{proof of closed curve theorem}
\pmrecord{6}{34164}
\pmprivacy{1}
\pmauthor{paolini}{1187}
\pmtype{Proof}
\pmcomment{trigger rebuild}
\pmclassification{msc}{30E20}

% this is the default PlanetMath preamble.  as your knowledge
% of TeX increases, you will probably want to edit this, but
% it should be fine as is for beginners.

% almost certainly you want these
\usepackage{amssymb}
\usepackage{amsmath}
\usepackage{amsfonts}

% used for TeXing text within eps files
%\usepackage{psfrag}
% need this for including graphics (\includegraphics)
%\usepackage{graphicx}
% for neatly defining theorems and propositions
%\usepackage{amsthm}
% making logically defined graphics
%%%\usepackage{xypic}

% there are many more packages, add them here as you need them

% define commands here
\begin{document}
Let
\[
  f(x+iy) = u(x,y) + i v(x,y).
\]
Hence we have
\[
  \int_C f(z)\, dz = \int_C \omega + i \int_C \eta 
\]
where $\omega$ and $\eta$ are the differential forms
\[
  \omega = u(x,y)\,dx - v(x,y)\, dy,\qquad
   \eta = v(x,y)\, dx + u(x,y)\, dy.
\]
Notice that by Cauchy-Riemann equations $\omega$ and $\eta$ are closed differential forms. Hence by the lemma on closed differential forms on a simply connected domain we get 
\[
  \int_{C_1} \omega = \int_{C_2} \omega,\quad
  \int_{C_1} \eta = \int_{C_2} \eta.
\]
and hence
\[
  \int_{C_1} f(z)\, dz = \int_{C_2} f(z)\, dz
\]
%%%%%
%%%%%
\end{document}
