\documentclass[12pt]{article}
\usepackage{pmmeta}
\pmcanonicalname{HyperbolicSineIntegral}
\pmcreated{2013-03-22 18:27:48}
\pmmodified{2013-03-22 18:27:48}
\pmowner{pahio}{2872}
\pmmodifier{pahio}{2872}
\pmtitle{hyperbolic sine integral}
\pmrecord{7}{41131}
\pmprivacy{1}
\pmauthor{pahio}{2872}
\pmtype{Definition}
\pmcomment{trigger rebuild}
\pmclassification{msc}{30A99}
\pmsynonym{Shi}{HyperbolicSineIntegral}
\pmrelated{HyperbolicFunctions}
\pmrelated{SineIntegral}

% this is the default PlanetMath preamble.  as your knowledge
% of TeX increases, you will probably want to edit this, but
% it should be fine as is for beginners.

% almost certainly you want these
\usepackage{amssymb}
\usepackage{amsmath}
\usepackage{amsfonts}

% used for TeXing text within eps files
%\usepackage{psfrag}
% need this for including graphics (\includegraphics)
%\usepackage{graphicx}
% for neatly defining theorems and propositions
 \usepackage{amsthm}
% making logically defined graphics
%%%\usepackage{xypic}

% there are many more packages, add them here as you need them

% define commands here
\DeclareMathOperator{\Shi}{Shi}
\DeclareMathOperator{\Si}{Si}
\theoremstyle{definition}
\newtheorem*{thmplain}{Theorem}

\begin{document}
\PMlinkescapeword{expansion}
The function {\em hyperbolic sine integral} (in Latin {\em sinus hyperbolicus integralis}) from $\mathbb{R}$ to $\mathbb{R}$ is defined as
\begin{align}
\Shi{x} \,:=\, \int_0^x\frac{\sinh t}{t}\,dt,
\end{align}
or alternatively as
$$\Shi{x} \,:=\, \int_0^1\frac{\sinh{tx}}{t}\,dt.$$\\

It isn't an elementary function.\, The equation (1) implies the Taylor series expansion
   $$\Shi{z} = z\!+\!\frac{z^3}{3\!\cdot\!3!}\!+\!\frac{z^5}{5\!\cdot\!5!}
                \!+\!\frac{z^7}{7\!\cdot\!7!}\!+\cdots,$$
which converges for all complex values $z$ and thus defines an entire transcendental function.\, 
Using the Taylor expansions, it is easily seen that
$$\Shi x \;=\; i\,\Si{ix}$$
connects Shi to the sine integral function.\\

$\Shi{x}$ satisfies the linear third \PMlinkescapetext{order} differential equation
          $$xf'''(x)\!+\!2f''(x)\!-\!xf'(x) = 0.$$

%%%%%
%%%%%
\end{document}
