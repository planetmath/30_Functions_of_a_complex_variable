\documentclass[12pt]{article}
\usepackage{pmmeta}
\pmcanonicalname{ExamplesOfPeriodicFunctions}
\pmcreated{2013-03-22 17:57:29}
\pmmodified{2013-03-22 17:57:29}
\pmowner{pahio}{2872}
\pmmodifier{pahio}{2872}
\pmtitle{examples of periodic functions}
\pmrecord{10}{40458}
\pmprivacy{1}
\pmauthor{pahio}{2872}
\pmtype{Example}
\pmcomment{trigger rebuild}
\pmclassification{msc}{30A99}
\pmclassification{msc}{26A09}
\pmsynonym{common periodic functions}{ExamplesOfPeriodicFunctions}
\pmrelated{PeriodicityOfExponentialFunction}
\pmrelated{HyperbolicIdentities}
\pmrelated{RationalAndIrrational}
\pmrelated{PeriodicFunctions}
\pmrelated{FloorFunction}
\pmrelated{Floor}

\endmetadata

% this is the default PlanetMath preamble.  as your knowledge
% of TeX increases, you will probably want to edit this, but
% it should be fine as is for beginners.

% almost certainly you want these
\usepackage{amssymb}
\usepackage{amsmath}
\usepackage{amsfonts}

% used for TeXing text within eps files
%\usepackage{psfrag}
% need this for including graphics (\includegraphics)
%\usepackage{graphicx}
% for neatly defining theorems and propositions
 \usepackage{amsthm}
% making logically defined graphics
%%%\usepackage{xypic}

% there are many more packages, add them here as you need them

% define commands here

\theoremstyle{definition}
\newtheorem*{thmplain}{Theorem}

\begin{document}
We list common periodic functions. In the parentheses, there are given their period with least modulus.

\begin{itemize}

\item One-periodic functions with a real period:

sine ($2\pi$), cosine ($2\pi$), tangent ($\pi$), cotangent ($\pi$), secant ($2\pi$), cosecant ($2\pi$), and functions depending on them -- especially the triangular-wave function ($2\pi$); \,the mantissa function
$x\!-\!\lfloor{x}\rfloor$ (1). 

\item One-periodic functions with an \PMlinkname{imaginary}{ImaginaryNumber} period:

exponential  function ($2i\pi$), hyperbolic sine ($2i\pi$), hyperbolic cosine ($2i\pi$), hyperbolic tangent ($i\pi$), hyperbolic cotangent ($i\pi$), and functions depending on them.

\item Two-periodic functions:\, elliptic functions.

\item Functions with \PMlinkname{infinitely}{Infinite} many periods:

the Dirichlet's function
\begin{eqnarray*}
 f\!:\; x\mapsto\! & \left\{ \begin {array}{ll} 1 & \mbox{when}\,\,x \in \mathbb{Q}\\
 0 & \mbox{when}\,\, x \in \mathbb{R}\!\smallsetminus\!\mathbb{Q}
 \end{array} \right.
 \end{eqnarray*}
has any rational number as its period;\, a constant function has any number as its period.

\end{itemize}
%%%%%
%%%%%
\end{document}
