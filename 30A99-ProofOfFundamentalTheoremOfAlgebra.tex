\documentclass[12pt]{article}
\usepackage{pmmeta}
\pmcanonicalname{ProofOfFundamentalTheoremOfAlgebra}
\pmcreated{2013-03-22 13:09:39}
\pmmodified{2013-03-22 13:09:39}
\pmowner{scanez}{1021}
\pmmodifier{scanez}{1021}
\pmtitle{proof of fundamental theorem of algebra}
\pmrecord{5}{33603}
\pmprivacy{1}
\pmauthor{scanez}{1021}
\pmtype{Proof}
\pmcomment{trigger rebuild}
\pmclassification{msc}{30A99}
\pmclassification{msc}{12D99}

\endmetadata

% this is the default PlanetMath preamble.  as your knowledge
% of TeX increases, you will probably want to edit this, but
% it should be fine as is for beginners.

% almost certainly you want these
\usepackage{amssymb}
\usepackage{amsmath}
\usepackage{amsfonts}
\usepackage{mathrsfs}

% used for TeXing text within eps files
%\usepackage{psfrag}
% need this for including graphics (\includegraphics)
%\usepackage{graphicx}
% for neatly defining theorems and propositions
%\usepackage{amsthm}
% making logically defined graphics
%%%\usepackage{xypic}

% there are many more packages, add them here as you need them

% define commands here
\begin{document}
If $f(x) \in \mathbb{C}[x]$ let $a$ be a root of $f(x)$ in some
extension of $\mathbb{C}$. Let $K$ be a Galois closure of
$\mathbb{C}(a)$ over $\mathbb{R}$ and set $G = \operatorname{Gal}(K/\mathbb{R})$.
Let $H$ be a Sylow 2-subgroup of $G$ and let $L = K^H$ (the fixed field of $H$ in $K$).
By the Fundamental Theorem of Galois Theory we have
$[L:\mathbb{R}] = [G:H]$, an odd number. We may write $L =
\mathbb{R}(b)$ for some $b \in L$, so the minimal polynomial
$m_{b,\mathbb{R}}(x)$ is irreducible over $\mathbb{R}$ and of odd
degree. That degree must be 1, and hence $L = \mathbb{R}$, which
means that $G = H$, a 2-group. Thus $G_1 = \operatorname{Gal}(K/\mathbb{C})$ is also a 2-group. If $G_1 \ne 1$ choose $G_2
\le G_1$ such that $[G_1:G_2] = 2$, and set $M = K^{G_2}$,
so that $[M:\mathbb{C}] = [G_1:G_2] = 2$. But any polynomial of
degree 2 over $\mathbb{C}$ has roots in $\mathbb{C}$ by the
quadratic formula, so such a field $M$ cannot exist. This
contradiction shows that $G_1 = 1$. Hence $K = \mathbb{C}$ and $a
\in \mathbb{C}$, completing the proof.
%%%%%
%%%%%
\end{document}
